\newcommand*{\titleContent}{“大共同体本位”与传统中国社会}
\newcommand*{\authorContent}{秦晖}
\newif\ifafourpaper
\afourpapertrue
\ifafourpaper
\documentclass[a4paper,12pt,punct=kaiming,fontset=none]{ctexart}
\usepackage[hmargin=1.5in,vmargin=1in]{geometry}
\title{\bfseries\titleContent \\[1em] \large\normalfont——兼论中国走向公民社会之路}
\else
\documentclass[punct=kaiming]{ctexart}
\usepackage[a6paper,hmargin=.6in,vmargin=.3in]{geometry}
\title{\Large\bfseries\titleContent \\[1em] \small\normalfont——兼论中国走向公民社会之路}
\fi

\author{\authorContent}
\date{}

\usepackage{xcolor}

\usepackage[
  pdftitle={\titleContent},
  pdfauthor={\authorContent},
  hyperfootnotes=false,
  colorlinks=true,
  urlcolor={.},
  linkcolor={.}
]{hyperref}

\setmainfont{Times New Roman}[Ligatures=Rare]

\newfontfamily\enkai{KaiTi_GB2312}
\setCJKfamilyfont{entimes}{Times New Roman}
\newcommand*{\entimes}{\CJKfamily{entimes}}

\setCJKmainfont{Songti SC}[
  BoldFont = PingFang SC,
  ItalicFont = KaiTi_GB2312,
  AlternateFont = {
    {"2014}{Noto Serif CJK SC},
    {`!}{Adobe Song Std}
  }
]
\setCJKmainfont{Noto Serif CJK SC}[CharRange={"30FB}]

\ltjdefcharrange{10}{"2013}
\ltjsetparameter{jacharrange={-10}}
\ltjsetparameter{xkanjiskip=.13\zw plus 1pt minus 1pt}

\usepackage[authoryear]{natbib}
% \setlength{\bibhang}{0em}

\newcommand*{\enumparen}[1]{\textnormal{(}\mbox{#1}\textnormal{)}}
\renewcommand{\labelenumi}{\enumparen{\theenumi}}

\usepackage{tikz}
\usepackage{float}

\begin{document}
\maketitle

\noindent{\small\textbf{编者按:}\enkai\textit{本文由编者以“独立中文笔会”上的版本\footnote{\url{https://www.chinesepen.org/blog/archives/130968}}为底稿,参考秦晖连载在《社会学研究》\,1998年第5期、1999年第3~4期上的同名文章,以及发表在《战略与管理》1999年第6期上的《传统中国社会的再认识》一文,校正而成。}}

五四“新文化运动”以反思、批判中国传统、通过文化启蒙实现中国文化的现代化重建为标帜。从那时以来整整八十年,对“中国传统”的事实判断(中国传统究竟是什么)和价值判断(否定还是肯定传统,以及全盘否定和全盘肯定之间的各种“保守”与“激进”立场)一直是中国思想界的主题。而前一判断则是后一判断的基础。就这一基础而言,过去八十年主要形成了两大认识范式,一是强调“生产方式”的“封建社会论”,它导致了以“土地革命”(消灭地主阶级的“民主革命”)为核心的“反传统”运动。一是强调宗法伦理、整体和谐与非个性化的“儒家文明”论,它引出了倡导个性解放的自由主义“反传统”运动和反对“西方个人主义”的传统复兴运动。

然而这两大认识范式看来都遇到了解释危机。作为历史,它们都无法解释何以同出“反传统”阵营、而且早期具有极端个性解放或激进自由主义色彩的五四极左翼(陈独秀、李大钊等)后来会发展出一种比“传统”更压迫个性更敌视自由的整体性极权倾向、并在其发展至极端的文革时期忽然迸发了向古代“法家”(商鞅、秦始皇等)认祖归宗的热情——显然,用“反传统过激”或“传统的影响”都难以解释这一切。作为现实,它们更无法解释当前改革中大陆上“西化”(民主、市场、自由、人权等因素)与“传统化”(国学热、宗族的复兴与乡镇企业中的家族化色彩等)同时发生的机制,尤其无法说清两者间的关系。

本文试图跳出这两种范式,从乡土社会而不是从“思想家”的作品中寻求对“中国传统”的再认识。本文认为,传统中国乡村社会既不是被租佃制严重分裂的两极社会,也不是和谐而自治的内聚性小共同体,而是大共同体本位的“伪个人主义”社会。与其他文明的传统社会相比,传统中国的小共同体性更弱;但这非因个性发达,而是因大共同体性亢进所致。它与法家或“儒表法里”的传统相连,形成一系列“伪现代化”现象。小共同体本位的西方传统社会在现代化起步时曾经过“公民与王权的联盟”之阶段,而中国的现代化则可能要以“公民与小共同体的联盟”为中介。

\currentpdfbookmark{小共同体本位论质疑}{questioning}
\section*{小共同体本位论质疑}

对于传统中国社会,尤其是被视为中国文化之根的传统乡土社会,目前流行的主要有两大解释理论:一为过去数十年意识形态支持的“租佃关系决定论”,这一理论把传统农村视为由土地租佃关系决定的地主—佃农两极社会。土地集中、主佃对立被视为农村一切社会关系乃至农村社会与国家之关系的基础;宗族关系、官民关系乃至两性关系和神人关系都被视为以主佃对立为核心的“封建”关系(此即把政权、族权、神权与夫权都归之于“封建地主制”的“四大绳索”论);赋税、利息、商业利润等资源分配形式都被视为“地租的分割”;而“土地兼并—农民战争”的叙述模式被用以解释历史上的“周期性”动乱;“中国租佃制与西方农奴制”则成为解释中西之别、尤其是解释中国何以“没有产生资本主义”的首要因素。

笔者前已指出(秦晖 等 1996),这种解释模式存在着严重问题,而另一种解释模式,我们可称之为“乡土和谐论”。它在1949年以前曾与“租佃关系决定论”互为论敌,而在这以后由于非学术原因它在大陆消失数十年,改革以后才在传承1949年以前学统和引进外部(港台及海外汉学)学理的基础上复兴。然而有趣的是:此时它已不以“租佃关系决定论”为论战对手,而成了从“新保守”到“后现代”的各种观点人士排拒“西化”的一种思想武器。这种解释把传统村落视为具有高度价值认同与道德内聚的小共同体,其中的人际关系,包括主佃关系、主雇关系、贫富关系、绅民关系、家(族)长与家(族)属关系都具有温情脉脉的和谐性质。在此种温情纽带之下的小共同体是高度自治的,国家政权的力量只延伸到县一级,县以下的传统乡村只靠习惯法与伦理来协调,国家很少干预。所谓的乡绅则被视为“植根于乡土伦理而体现社区自治”的小共同体人格化身,绅权制衡着皇权(国家权力)的下伸意向。而此种伦理与自治的基础则是据说集中代表了中国独特文化并自古传承下来的宗族血缘纽带,因而传统乡村又被认为是家族本位的(并以此有别于“西方传统”的个人本位)。而儒家学说便是这种现实的反映,它以“家”拟“国”,实现了家国一体、礼法一体、君父一体、忠孝一体。于是儒家又被视为“中国文化”即中国人思维方式及行为规则的体现,它所主张的性善论、教化论、贤人政治、伦理中心主义等则被看作是中国特色之源。

从这套解释出发产生了各种各样的引伸:有人从这种“乡村自治传统”看到了中国的“小政府大社会”和中国传统比西方更“自由主义”(盛洪 1999)。有人则相反,从乡土中国小共同体的“集体主义”中看到了克服“西方现代性”的自由主义之弊的希望,并期待“乡土中国的重建”会把人类引入“后现代”佳境(甘阳 1994)。有人以根据这种小共同体伦理自治说创作出来的“山杠爷”之类形象为论据证明“西方的”法治不适用于中国,我们的社会秩序只能指望“本土文化资源”培育的伦理权威(朱苏力 1998)。还有人以“村落传统”说来解释人民公社,认为“温情脉脉的自然村落”是中国传统“长期延续的关键”,人民公社制度体现了突破这一传统而走向“现代化”的努力,并付出了当然的“代价”;但 1958 年的“大公社”对“村落”的破坏“过分”了,引起了灾难,后来的“队为基础”则向“村落传统”作出了让步,因而使公社得以正常运作二十年云云(张乐天 1998)。

总之,强调村落、家族(宗族)等小共同体的自治(相对于国家的干预而言)与和谐(相对于内部的分化而言)并将其视为不同于“异文化”的华夏文明特性所在,是这些看法的共同点。这种“小共同体本位”的中国传统观具有多元的学理渊源,带着新儒家色彩的中国学者(如民国年间的乡村建设派人士)的工作、日本学者平野义太郎、清水盛光等人战前提出的“乡村共同体”论以及M・韦伯这样的西方思想家关于中国官僚制能力有限的说法(韦伯 1995,133–144),都是这种传统观的来源之一。但这里又大致可分为两种情况:一种是重点强调小共同体的内部和谐,以否定那种突出村内阶级分裂的说法;另一种是重点强调小共同体的自治,以否定那种批判东方帝国“全能专制主义”的观点。前一种以民国时期的乡村建设派为代表,他们不赞成共产党人的农村阶级斗争学说,但也反对国民党的专制统治,因此极力抨击国家“不良政治”对乡村的压迫,而并不把当时的乡村看成“自治”的乐园(梁漱溟 1990,141–658)。后一种则以平野义太郎等战前日本人士为典型,他们强调中国传统“乡村共同体”的自治性是为了否认中国国家对农村的统治能力,而对村庄内部关系是紧张还是和谐并无特别兴趣。正如美国学者杜赞奇所指出的(杜赞奇 1994,195–196),这种观点是为当时日本军阀的“东亚共荣”论服务的。而在战后这种“将(小)共同体概念强加于中国乡村的作法”已被许多日本学者尤其是自由派与左派学者所否定。然而有趣的是,近年在国内流行的种种小共同体本位的传统观最强调的恰恰正是自治性而非和谐性,亦即更近似于平野义太郎等人的观点而非乡村建设派观点,尽管这些论者喜欢征引的是后者而非前者。我们的这些论者乐于在清算“批评东方专制的西方话语”的背景下重新发现“自治的乡土中国”,却并无当年乡村建设派那种与“租佃关系决定论”论战的兴趣。其中的一些作品还把“温情脉脉的传统村落”之说与关于土地改革的革命理论融于一书,似乎并不觉得二者存在着扞格(张乐天 1998,21–46)。

应当说,这种“小共同体本位的传统观”之复兴有相当的合理性。首先,若与1949年以后的农村体制相比,传统中国国家政权对农村社区生活的控制能力确实弱得多;而与历史上的王朝强盛期相比,小共同体本位论者所集中考察的晚清。民国又是个未世、乱世,所谓乱世英雄起四方,有枪便是草头王,国家对基层社会的控制能力也未必能达到强盛王朝的水平。说这个时期乡村小共同体是“自治”的也还讲得过去,其次如果抽象地谈村落、家族的小共同体凝聚力也不是不可以,任何时代人们对自己所在的群体有所依附都是可以设想的。与改革前的“唯阶级关系论”相比,如今谈论对家族、村落的认同至少是看到了传统社会中人际关系的多样性,这自然是个进步。

但在文化形态论的意义上讲传统中国的小共同体本位、把它视为区别于异文化的中国特征,并用它来作为解释历史与现实的主要基础,则是很可质疑的。首先,“乡村和谐论”比“租佃关系决定论”更无法解释中国历史上最突出的现象,即过去称为“农民战争”的周期性超大规模社会冲突。以地主与佃农的矛盾来解释“农民战争”本已十分牵强,没有任何证据表明租佃关系的发达与否与“农民战争”有关。而历代“农民战争”不仅极少提出土地要求,甚至连抗租减租都没有提出,却经常出现“不当差,不纳粮”“三年免征”“无向辽东死,斩头何所伤”之类反抗“国家能力”的宣传,以及“王侯将相宁有种乎”“苍天已死,黄天当立”“虎贲三千,直抵幽燕之地,龙飞九五,重开大宋之天”这类改朝换代的号召。而像《水浒》中描写的那种庄主率领庄客(即“地主”率领“佃农”)造国家的反的场面,在历史上也屡见不鲜,这就更难用“租佃关系决定论”解释了。但是,如果我们相信传统乡村和谐、乡村自治之说,就更无法理解这类历史现象。如果传统乡村的内部关系真是那样温情脉脉,而乡村外部的国家权力又只能达到县一级而无法干预乡村生活,那种社会大爆炸怎么可能发生?退一步讲,即便乡村内部关系存在着紧张,如果真是社区自治,国家权力无法涉及,爆炸又怎能突破社区范围而在全国水平上发生?更何况中国历史上的社会爆炸通常根本不是在社区内发生然后蔓延扩散到社区外,而是一开始就在“国家”与“民间社会”之间爆发,然后再向社区渗透的。汉之黄巾“三十六方同日而起”至为典型。

历史上但凡小共同体发达的社会,共同体内部矛盾极少能扩展成社会爆炸,在村社、采邑或札德鲁加(家族公社)活跃的前近代欧洲,农民与他们的领主发生的冲突如果在小共同体内不能调解,也只会出现要求国家权力出面调解的现象而不是推翻国家权力(王权)的现象。像英国的Wat Tyler 之变、法国的 Jacques 之变这些“农民起义”实际上都不过是向国王进行武装请愿而已。俄国村社农民的“皇权主义”更是著名。即使在中国,明清之际租佃制最发达而宗族组织也相对更活跃的江南地区频频出现的“佃变”“奴变”也多采取向官府请愿的方式,而与北方自耕农及破产自耕农(流民)为主体的改朝换代的“农民战争”截然有别。于是地主—佃农矛盾最为突出的这个地区反而成为席卷全国的明末“农民战争”锋头所不及的“偏安”之地。过去人们基于某种理论往往乐于设想:地主与农民发生租佃或土地纠纷,而官府出面支持地主镇压农民,使民间的贫富矛盾膨胀为官民矛盾,于是引发大乱。这种事例当然不是没有(前引的江南佃变即为其例)。但中国历史上更为习见的却是相反:因国家权力的横征暴敛、取民无度,或滥兴事业、役民无时,或垄断利源、夺民生计,或吏治败坏、虐民无休,而引发官民冲突;故俗语历来有“官逼民反”而从无“主逼佃反”之说。

官民矛盾一旦激化,民间贫富态度因之生异:一般民间有声望者多富,出头抗官者亦多富民;但就从者而言,则贫苦者穷则思变、有身家者厌乱思安,于是官民冲突扩及民间而引发贫富对立。

例如明末大乱,本因“天灾、加派、裁驿、逃军”而起。这四项本与租佃关系无涉,除“天灾”外就是官民矛盾。而其中最致命的“加派”按官方的意图甚至主要是针对富民(不是富官,也非穷民)的:“计亩而征”、“弗以累贫不能自存者,素封是诛”([康熙]《河南通志》卷四十)、“殷实者不胜诛求之苛”([顺治]《鄢陵县志》)。于是社会上出现的也不是什么土地兼并,而是“村野愚懦之民以有田为祸”(《西园闻见录》赋役前、袁表语)、“至欲以地白付人而莫可推”、“地之价贱者亩不过一二钱,其无价送人而不受者大半”([康熙]《三水县志》卷四)。在这种情况下爆发的“农民战争”,早期是一种“流寇”(破产自耕农汇聚的流民武装)与“土寇”(聚“庄佃”而抗官的富民庄主)一起造朝廷的反的“土流并起”之局([康熙]《上蔡县志》卷十二)。但到明亡前夕,官府统治已解体,满地“流寇”的“乱世”已成了庄主们面临的主要威胁。于是便出现了这样的现象:当明朝力量尚强时勇于反抗的“土寇”,到明朝已无力镇压时反而纷纷“就抚”,成为朝廷眼中“介于似贼似民之间”(秩名:《晴雪斋漫录》卷四)的力量。而“流寇”与庄主的冲突便开始激化了——但即使在这种情况下,“流土冲突”虽可以说是贫富冲突,却依然难说是主佃冲突,因为流土之间并无主佃关系,“流寇”并不起源于佃农,而此时的“庄佃”随庄主对抗“流寇”,与此前他们随庄主抗官本无二致。

并非民间贫富冲突而使官府卷入,而是官逼民反导致民间贫富冲突。这种中国独有的“农民战争”机制是“租佃关系决定论”和“乡村和谐论”都不能理解的。

\currentpdfbookmark{“道德农民”与“理性农民”之外}{peasant}
\section*{“道德农民”与“理性农民”之外}

谈到共同体,不能不提及70年代国外社会学及文化人类学界影响很大的一场争论。当时由于美国在印度支那的失败,引起了研究越南(以及东南亚)农村社会(其“问题意识”显然源于“当地农民为什么支持共产党”)的热潮。这一热潮在学理上又受到50年代M・马略特和R・列费尔德等人在农民研究中强调“小共同体(或译小社区)”与“小传统”(区别于民族或国家的“大传统”)之重要性的影响(Marriot 1955; Redfield 1955, 1956)。1976年,J・C・斯科特发表《农民道德经济:东南亚的叛乱与生计》一书,提出了强调“亚洲传统集体主义价值”的“道德农民”理论。据他所说,亚洲农民传统上认同小共同体,全体农民的利益高于个人权利;社区习惯法的“小传统”常常通过重新分配富人的财产来维护集体的生存。这种与“西方个人主义”不同的价值在他看来,便是美国失败的深层文化原因(Scott 1976)。

斯科特的看法引起了商榷, S・波普金1979年发表的《理性的农民:越南农村社会政治经济学》在反对声中最为典型。波普金认为越南农民是理性的个人主义者,由他们组成的村落只是空间上的概念而并无利益上的认同纽带;各农户在松散而开放的村庄中相互竞争并追求利益最大化。尽管他们偶尔也会照顾村邻或全村的利益,但在一般情况下各家各户都是自行其是自谋其利的(Popkin 1979)。在后来的讨论中支持波普金者不乏其人,有人甚至称:“亚洲的农民比欧洲的农民更自私”。

无疑,这样的争论也可以发生在中国农民研究者中:中国农民在本质上是“道德农民”还是“理性农民”?是休戚与共的小共同体成员还是仅仅居住在同一地域上的一群自私者?可能的回答是“两重性”之说,即他们同时具有这两种性格。由此又可能导出哪一种为主(例如说贫农以“道德”为主而富农以“理性”为主等等)之争。

然而更大的问题在于:如果说“道德农民”与“理性农民”是不相容的,那么这二者的不相容能推导出它们的“逆相容”吗?即:如果农民社会缺少“理性”(特指“个人理性”),这就意味着“小共同体的道德自治”吗?反之,如果农民对社区、宗族缺少依附感,他就一定是个性发达的“便士资本家”或理性至上的个人主义者吗? 这些却是斯科特与波普金都未想过的。

欧洲人的这种态度并不难理解。西欧在个体本位的近代公民社会之前是“小共同体本位”的社会,人们普遍作为共同体成员依附于村社(马尔克)、采邑、教区、行会或家族公社(南欧的扎德鲁加),而东欧的俄罗斯传统农民则是米尔公社社员。近代化过程是他们摆脱对小共同体的依附而取得独立人格、个性自由与个人权利的过程。这些民族的传统社会的确有较发达的社区自治与村社功能:小共同体的“制度性传统”相当明显。如俄罗斯的米尔有土地公有、定期重分之制,有劳动组合及“共耕地”之设,实行“征税对社不对户,贫户所欠富户补”的连环保制度;为三圃制与敞地放牧的需要,村社还有统一轮作安排、统一农事日程的惯例。村社不仅有公仓、公牧、公匠,还有村会审判与村社选举等“小共同体政治”功能。农民离村外出首先要经村社许可,其次才是征得领主与官厅的批准。农户的各种交纳中,对领主的负担占54\%,对国家的负担占19.8\%,而对村社的负担要占到26\%(金雁 等 1996);甚至住宅也必须建在一处,“独立农庄”只是近代资本主义改革后才少量地出现。

西欧的马尔克虽然没有俄国米尔那样浓厚的“经济共同体”色彩,但农用地仍有村社“份地”名分,并实行敞地制、有公共林牧地与磨坊。尤其因中古西欧没有俄罗斯那样的集权国家,其村社与采邑的“政治共同体”色彩比俄国米尔更浓郁,村社习惯法审判也比俄国更活跃。像电影《被告山杠爷》那样的耆老断事惯例,在中古欧洲恐怕要比在中国典型得多。

甚至在斯科特与波普金之争所关注的越南,“小共同体”的活跃程度可能也在传统中国之上。据当代社会人类学研究,在19~20世纪之交的越南农村,制度化公共生活仍颇为可观。在传世村社文书中有许多关于农民家内矛盾的案卷;有趣的是那里个体家庭中的父权与夫权似不如传统中国凸显,但社区伦理干预却超过中国。在旧中国,卖儿卖女可能触犯“国法”,但“社区习惯”却很少管这种事;而在这些案卷中家长出卖子女却会引起村社“耆目会同”(长老会)的干预。这种“会同”还会为女性要求继承权、要求离婚男子赔偿女方等等。据说那里的公田要占田地总量1/3,村社财政十分活跃。尤为令人惊奇的是那里存在着跨自然村的社区自治组织:数屯(自然村)为一“社”,数“社”为一“总”。“社”之设由各屯自发组合,并非官方区划,亦无一定规模。“社”平均2000人左右,有大至3~4万者。“社”首称“里役”,有里长、副里长、先祗等,皆由村民以“具名投票”选出,但多有宗族背景(往往一届里役同出一宗),任期三年,政府不干预其人选。卸任里役多进入“耆目会同”:据某县之耆目名单,其中80\%为前里役,1\%为退休官吏,19\%为其他当选者,而与文献(乡约)中规定耆目应由儒士中选出不同。在19世纪前“耆目会同”权重于里长,可决定公田分配等要务,故里长多求卸任为耆目而不愿连任;入20世纪后里长权始渐重,亦乐求连任矣。但无论里长还是耆目均属民间人物,知县并非其上级,亦不干涉社内事务。社实行司法自治,遇有案件知县也会下到社里与里长共同调查,但只起调解人作用,并不能左右里长。

这样的社区自治是古来传统使然,还是殖民时代法国人影响的结果?P・帕潘认为是前者。但也有记载说,15~16世纪的里长为政府任命,并非民选的。或许正如本世纪前半叶 J. H. Boeke 及J. Furnivall 等人研究的爪哇、缅甸“传统”村社后来被指出是殖民时代发展的社区组织一样,上述的越南村社也并非原生形态的传统共同体。但无论如何,这样的“小共同体”与“乡村自治”已超过了我们所知的传统中国多数地区,尤其是更传统、受近代化影响更小的北方内地农村的情况。

长期在北方倡导“乡村建设”的梁漱溟先生的“伦理本位”论对今日谈“村落传统”者很有影响,但正是梁漱溟对北方传统农村缺乏“小共同体”认同和村社组织深有感触。他说:“中国人切己的便是身家,远大的便是天下了。小起来甚小,大起来甚大……。西洋人不然。他们小不至身家,大不至天下,得乎其中,有一适当范围,正好培养团体生活”(梁漱溟 1990,194)。的确如此,像传统欧洲那样的村社、采邑、教区、行会和家族,中国传统中是缺少的;像俄罗斯、印度乃至越南那样的“小共同体”,中国古时也难见到。中国古时也有土地还授之制,但那不是村社而是国家行为(名田、占田、均田、计口授田以及旗地制等);中国农民历史上也有迁徒限制,但那不是米尔而是国家对“编户齐民”的管束;中国人知道朝廷颁布的“什伍连坐”之法,但不知道何谓“村社连环保”;中国农民知道给私人地主交租,给朝廷交皇粮国税,却不会理解向“小共同体”交纳26\%是怎么回事。当然,中国人知道家规族法与祠堂审判,“文化”爱好者把这种“习惯法与伦理秩序”设想为原生态的“本土”现象而在闭塞落后的中西部“山”中设计“杠爷”形象。然而现实中的“杠爷”却集中发生在更受“西方”影响的东南沿海地区,内地闭塞的臣民们反而更懂得“王法”而不知道什么叫村社审判。

我国近年来在改革中一些农村出现了活跃的“庄主经济”,同时在东南发达地区出现了“宗族复兴”。对此,有人惊为“封建的沉渣泛起”,有人则褒之为“传统文化的伟大活力”。其实,“封建”也好,“传统”也罢,它们与“宗族”的关系究竟如何是大可反思的。我国民间的“非宗族”现象其源甚古:上古户籍资料如江陵凤凰山出土的西汉《郑里廪簿》与郸县犀浦出土的汉代《赀簿》残碑都呈现出突出的多姓杂居,其中犀浦残碑涉及18户,能辨出姓名的11个户主中就有至少六七个姓(秦晖 1987);敦煌文书中的唐代儒风坊西巷社34户中亦有12俗姓外加3僧户,另一“社条”所列13户中就有9个姓。而至少在宋元以后,宗族的兴盛程度出现了与通常的逻辑推论相反的趋势:越是闭塞、不发达、自然经济的古老传统所在,宗族越不活跃;而是越外向、商品关系发达的后起之区反而多宗族。从时间看,明清甚于宋元;从空间看,东南沿海甚于长江流域,长江流域又甚于黄河流域。直到近代,我们还看到许多经济落后、风气闭塞的北方农村遍布多姓杂居村落,宗族关系淡漠;而不少南方沿海农村却多独姓聚居村,祠堂林立,族规森严,族谱盛行。内地的传统发祥之区关中各县土改前族庙公产大多不到总地产的1\%,而广东珠江三角洲各县常达30~50\%乃至更高。浙江浦江县全县地产1/3为祠庙公产,义乌县有的村庄竟达到80\%(秦晖 等 1996,168)。这种现象究竟意味着什么,是耐人寻味的。

要之,“传统”的中国社会并不像人们通常想象的那样以宗族为本。而宗族以外的地缘组织,从秦汉的乡亭里、北朝的邻里党直到民国的保甲,都是一种官方对“编户齐民”的编制。宗族、保甲之外,传统中国社会的“小共同体”还有世俗或宗教的民间秘密会社,以及敦煌文书中显示的民间互助组织“社邑”(主要指“私社”)。但秘密会社是非法的,而“社邑”的功能仅限于因事而兴的丧葬互助等有限领域,与村社不可同日而语。因此如果说就村社传统之欠缺而言,越南农民比欧洲农民更“自私”,那么说传统中国农民比越南农民更“自私”大约也不错。可见,如果说“道德经济”是指农民对村社或“小共同体”的依附,那么这一概念不适用于中国农民,至少不像适用于俄国或西欧农民那样适用于中国农民。但这是否意味着中国农民就更像是“个人主义的理性农民”或“便土资本家”?

这就涉及现代中国史上那个令人难解的“公社之谜”。当年苏联发动集体化时,斯大林曾把俄国的村社土地公有传统视为集体化可行的最重要依据,他宣称恩格斯在农民改造问题上过于慎重,是由于西欧农民有小土地私有制。而俄国没有这种东西,因此集体化能够“比较容易和比较迅速地发展”。而中国农民的“小私有”传统似乎比西欧农民还要悠久顽强,因此50年代中国集体化时,许多苏联人都认为不可行。然而事实表明,中国农民虽然并不喜欢集体化,但也并未表现出捍卫“小私有”的意志。当年苏联为了把(土地)“公有私耕”的村社改造为公有公耕的集体农庄付出了惨重的代价:逮捕、流放了上百万的“富农”,为镇压农民反抗出动过成师的正规军和飞机大炮,而卷入反抗的暴动农民仅在1930年初就达70万人。“全盘集体化运动”费时四年,而在农民被迫进入集体农庄时,他们杀掉了半数以上的牲畜(沈志华 1994,422–432)。而中国农民进入人民公社只花了短得多的时间,也未出现普遍的反抗。

为什么“小私有”的中国农民比俄国的村社农民更易于被集体化?我们看看俄国村社在这段转折中的作用就会明白。俄国传统村社作为前近代小共同体对农民个性与农民农场经济的发展有阻碍作用,这是使包括斯大林在内的一些人认为它有利于集体化的原因。事实上在新经济政策后期政府也的确利用村社来限制“自发势力”,抑制“独立农庄化”倾向。但村社作为小共同体的自治性又使它具有制衡“大共同体”的一定能力。在集体化高潮前夕,传媒曾惊呼农村中出现了米尔与村苏维埃“两个政权并存”的局面,并报道了许多“富农”(当时实际上指集体化的反对者)把持村社的案例,如卢多尔瓦伊事件、尤西吉事件等。显然,具有一定自治性的村社是使俄国农民有组织地抵制集体化的条件,因此苏联在集体化高潮的1930年宣布废除村社就毫不奇怪了。而中国革命后形成的是农户农村。本来就不如俄国村社那么强固的传统家族、社区的小共同体纽带也在革命中扫荡几尽,连革命中产生的农会在土改后也消亡了。农村组织前所未有地一元化,缺乏可以制衡大共同体的自治机制,于是“小私有”的中国农民反而比“土地公有”的俄国村社更易于“集体化”就不难理解了。

这其实也体现了一种中国传统。受“村社解体产生私有制”的理论影响,长期以来我国史学界那些否认中国古代有过“自由私有制”的人总是强调小共同体的限制因素,如土地买卖中的“亲族邻里优先权”和遗产遗嘱中的“合族甘结”之类。但实证研究表明,这样的限制在中国传统中其实甚弱,中国的“小农”抗御这种限制的能力,要比例如俄国农民抗御村社限制的能力大得多。但同时这些缺乏自治纽带的“小农”对大共同体的制御能力却很差。因缺乏村社传统似乎更为“私有”化的中国农民,反而更易受制于国家的土地统制,如曹魏屯田、西晋占田、北朝隋唐均田、北宋的“西城刮田”与南宋“公田”。明初“籍诸豪民田以为官田”以致“苏州一府无虑皆官田,民田不过十五分之一”,直到清初的圈占旗地等等。

于是传统中国农民便很大程度上置身于“道德农民”与“理性农民”之外:小共同体在这里不够发育,但这并非意味着个性的发育,而是“大共同体”的膨胀之结果。而这一传统就说来话长了。

\currentpdfbookmark{法家传统与大共同体本位}{legalism}
\section*{法家传统与大共同体本位}

中国的大一统始于秦,而关于奠定了强秦之基的商鞅变法,过去史学界有个标准的论点,即商鞅坏井田、开阡陌而推行了“土地私有制”。如今史学界仍坚持此种说法的人怕已不多,因为70年代以来人们从睡虎地出土秦简与青川出土的秦牍中已明确知道秦朝实行的是严格的国家授地制而不是什么“土地自由买卖”;而人们从《商君书》、《韩非子》一类文献中也不难发现秦代法家经济政策的目标是“利出一孔”的国家垄断,而不是民间的竞争。

然而过去人们的那种印象却也非仅空穴来风。法家政策的另一面是反宗法、抑族权、消解小共同体,使专制皇权能直接延伸到臣民个人而不致受到自治团体之阻隔。因此法家在理论上崇奉性恶论,黜亲情而尚权势,公然宣称“夫以妻之近及子之亲而犹不可信,则其余无可信者矣”(《韩非子・备内》)。在实践上则崇刑废德,扬忠抑孝,强制分家,鼓励“告亲”,禁止“容隐”,不一而足。尤其有趣的是,出土《秦律》中一方面体现了土地国有制,一方面又为反宗法而大倡个人财产权,给人以极“现代”的感觉。《秦律》中竟然有关于“子盗父母”“父盗子”“假父(义父)盗假子”的条文,并公然称:奴婢偷盗主人的父母,不算偷了主人;丈夫犯法,妻子若告发他,妻子的财产可以不予没收;而若是妻有罪,丈夫告发,则妻子的财产可用于奖励丈夫。即一家之内父母子女夫妻可有各自独立的个人财产。于是乎便出现了这样的世风:“借父耰鉏,虑有德色;母取箕帚,立而谇语;抱哺其子,与公并踞;妇姑不相悦,则反唇相讥”。这里亲情之淡漠,恐怕比据说父亲到儿子家吃饭要付钱的“西方风俗”犹有过之!难怪人们会有商鞅推行“私有制”的印象了。

然而正是在这种“爹亲娘亲不如皇上亲”的反宗法气氛下,大共同体的汲取能力可以膨胀得漫无边际。秦王朝动员资源的能力实足惊人:两千万人口的国家,北筑长城役用40万人,南戍五岭50万人,修建始皇陵和阿房宫各用(一说共用)70余万人,还有那工程浩大的驰道网、规模惊人的徐福船队……这当然不是“国家权力只达到县一级”所能实现的。其实按人口论,秦时之县不比今日之乡大多少,秦时达到县一级已相当今日达到乡一级了。然而秦县以下置吏尚多。“汉承秦制”,我们可以从汉制略见一斑。“大率十里一亭,亭有长;十亭一乡”,“乡有三老:有秩、啬夫、游徼。三老掌教化:啬夫职听讼、发赋税,游徼徼循、禁盗贼”,“又有乡佐,属乡,主民收赋税。”这些乡官有的史籍明载是“常员”,由政府任命并以财政供养:“有秩,郡所署,秩百石,掌一乡人”,有的则以“复勿徭戍”为报酬。所不同者,县以上官吏由朝廷任命(“国家权力只到县一级”仅在这个意义上才是对的),而这些乡官则分别由郡、县、乡当局任命。但他们并非民间自治代表则是肯定的。

秦开创了大共同体一元化统治和压抑小共同体的法家传统。小共同体解体导致的“私有制”看来似乎十分“现代”,但这只是“伪现代”,因为这里小共同体的解体并非由公民个人权利的成长,而是相反地由大共同体的膨胀所致。而大共同体的膨胀既然连小共同体的存在都不容,就更无公民权利生长的余地了。所以这种“反宗法”的意义与现代是相反的。宗族文化与族权意识在法家传统下自无从谈起,然而秦人并不因此拥有了公民个人权利。相反,“暴秦苛政”对人性、人的尊严与权利的摧残,比宗族文化兴盛的近代东南地区更厉害。

汉武帝改宗儒学,弘扬礼教,似乎是中国传统的一大转折。然而,“汉承秦制”且不说,“汉承秦法”尤值得重视。正如瞿同祖先生早已指出:武帝以后之汉法仍依秦统,反宗法的大共同体一元化色彩甚浓。而“儒家有系统之修改法律则自曹魏始”(瞿同祖 1981,334)。由魏而唐,中国的法律发生了个急转弯:以礼入法,礼法合一。法律儒家化实际上是社会上共同体多元化的反映。宗族兴起,族权坐大,小共同体的兴盛成为一时潮流,从魏晋士族一直发展到“百室合户、千丁共籍”的宗主督护制。社会精英主流也由秦汉时为皇上六亲不认的法家之吏变成了具有小共同体自治色彩、以“德高望重”被地方上举荐的“孝廉”“贤良方正”之属,并发展为宗法色彩极浓的门阀土族。这可以说是中国历史上一个罕见的“表里皆儒”的时代,然而值得注意的是:这也正是一个大一统帝国解体,类似于领主林立的时代。

从北魏废宗主督护而立三长始直到唐宋帝国复兴,中国出现了“儒表法里”的趋势并在此基础上重建了大共同体一元化传统,此一传统基本延续到明清。

“儒表法里”即在表面上承认多元共同体权威(同等尊崇皇权、族权、父权、绅权等等),而实际上独尊一元化的大共同体;讲的是性善论,信的是性恶论;口头的伦理中心主义,实际的权力中心主义;表面上是吏的儒化,而实质上是儒的吏化。在社会组织上,则是表面上崇尚大家族而实际效果类似“民有二男不分异者倍其赋”。

由隋至宋臻于完善的科举制是这一时期“儒表法里”的一大制度创新。从科举考试的内容看,它似乎有明显的儒家色彩,然而朱熹这样的大儒却对此制十分不满。其实这一制度本身应当说主要是法家传统的体现。事实上,更能体现儒家性善论与宗法伦理的选官制度应当是有点贵族政治色彩的、由道德偶像式的地方元老举荐“孝廉”“贤良方正”为官的察举之制——明儒黄宗羲正是主张用这类制度取代科举的。科场的严密防范以人性恶为前提,而识者已指出:设计巧妙的八股程式与其说是道德考试不如说是智力测验。唐太宗的名言“天下英雄(不是天下贤良!)入吾彀中”更说明这一制度的目的在于大一统国家通过“不知亲疏远近贵贱美恶,一以度量断之”的法家原则(《管子・任法》)把能人(而非贤人)垄断于掌握之中,它与一以耕战之功利择吏的秦法主要是所测之能不同而己。实际上由察举、门阀之制向科考之制的演变在某种程度上是对由周之世卿世禄到秦之军功爵制度的一种复制。儒家贵族政治被废弃并代之以“冷冰冰的”科场角逐,无疑是极权国家权威对宗法权威、“法术势”对温情主义占优势的结果。

近年来以科举制类比现代文官制度之风甚盛,其实这就像村社传统欠缺时的“私有制”在大共同体本位条件下成为一种“伪现代化”一样,贵族政治传统欠缺时的科举制在大共同体本位下也是“伪现代”性的。正如识者所云:科举官僚制的发展与其归之为社会上公共事务增多和分工发展的结果,倒不如更直接地理解为专制统治越益过度或无谓地分割官僚权任,又要保证一种更为集中的一元化控制秩序的产物。……这就是我国的近代化过程所以始终无法将它嫁接到共和体制上,及其所以在近代与帝制同归于尽的很大一部分原因(楼劲 等 1992,3)。通过科举制实现了表面上吏的儒化和实质上儒的吏化。近人常把科举制下的乡绅视为社区自治的体现者,实际上科举制以前的地方贵族倒庶几有点自治色彩,后来的乡绅更谈不上了。

这一时期的法律体系仍然保持魏晋以来的礼法合一性质。但维持小共同体的、宗法式的内容逐渐虚化,而维护大共同体的、一元化的内容逐渐实化。成文法形实分离的趋势,从宋律对唐律中过时和无意义的内容(如关于均田制与租庸调方面的内容)也全盘照抄即可见一般。一些维护大家族、宗族制的律文,如“诸祖父母、父母在而子孙别籍异财者徒三年”“民四十以上无子方听纳妾”等等,与现实已相去甚远。如果相信律文,中国应当是个典型的大家族社会,但实际上中西人口史、家庭史的资料表明:这一时期中国人的平均家庭规模小于西方,更重要的是如前所说,家庭之上的小共同体纽带更比西方松散。然而另一方面,如明初朱元璋的《大诰》等文献所显示的那样,那些维护皇权、维护大共同体一元化的律条,却不但名实相符,而且还有法外加酷、越律用刑的发展趋势。

汉以后历代统治者改宗儒学后,弘扬礼教,褒奖大家族,“大共同体”与“小共同体”的关系形式上比秦较为和谐。然而实际上法家传统一直存在,由汉到清的统治精神(除了前述魏晋以后一个时期外)仍然是“大共同体本位”的,而不是小共同体本位,更不是个人本位的。像古希腊的德莫,古罗马的父权制大家族,中世纪西欧的村社、行会、教区,俄罗斯的米尔等等这类含有自治因素的“非国家”社群所享有的地位,在传统中国很难想象。北宋是我国历史上一个较为宽松的时代,朝廷对民间共同体还是盯得很紧,即使是由政府号召成立的也不例外。元祐年间朝廷号召团结乡兵,苏轼就这样指出了两种乡兵类型:“陕西河东弓箭手,官给良田,以备甲马。今河朔沿边弓箭社,皆是人户祖业田产,官无丝毫之损”。如此看来,河朔弓箭社不是比陕西弓箭手更可取么?但恰恰相反。因为陕西乡兵完全由有司严密控制,从队、将直到提举司形成了严格的科层组织,虽不领饷,却完全是官办团体。而河朔弓箭社却具有太浓的民间色彩:“百姓自相团结为弓箭社,不论家业高下,户出一人。又自相推择家资武艺众所服者为社头、社副、录事,谓之头目”,“私立赏罚,严于官府”。这就足以使人害怕:“弓箭社一切兵器,民皆自藏于家,不几于借寇哉?”结论自然是:不许。

\currentpdfbookmark{“拜占廷现象”与“反宗法的非公民社会”}{byzantine}
\section*{“拜占廷现象”与“反宗法的非公民社会”}

文化类型学的研究者往往把家族本位视为“中国传统文化”的特征而以之与西方的“个人本位文化”相比较,这种看法在近代中国家族兴盛和西方个性解放的背景下或许不无道理。然而搬用到历史上却远非都是适宜的。我们且不说欧洲中世纪,那时宗族血亲关系与封主一封臣间的政治依附关系构成互为表里的两种基本人际纽带,而且前者的重要性达到如此程度,以致“除了由血缘纽带联结的人际关系外,不存在真正的朋友关系”(Bloch 1962, 2:124);那时西欧的宗族械斗,宗族仇杀、经济上的宗族公产及宗族对个人产权的干预与限制、族权对宗族成员的束缚与庇护,乃至数代同堂共炊合食的大家庭之常见,都很难说亚于古代中国。尽管非血缘性的村社、教区与封主——封臣依附纽带更多地被今人提到,但这些非血缘纽带在社会关系中对血缘共同体的优势是否比中国的专制国家统辖“编户齐民”的能力在社会关系中对宗族纽带的优势更大,是非常值得怀疑的。正因为如此,当代许多欧洲学者都把中世纪欧洲向近代欧洲的演变,称为“从宗族社会到公民社会”(From Lineage Society to Civil Society)(James 1974, 177–198)。就像我们形容中国“传统社会”的所谓“西化”一样。其实在笔者看来,如果不是把眼光局限在人类学家喜欢用作“文化标本”的若干村庄(往往是东南沿海地区的近代村庄)而是从大的时空尺度看,古代中国的基层社会组织是决不比中古欧洲更有资格叫做 lineage society 的。

如今的人们讲“西方传统”往往跳过中世纪而直接从希腊罗马寻找西方之“根”。“罗马法中的个人主义”与“罗马法意义上的私有财产”成为最常被提到的因素。可是人们却常常忽视:我们今天所见的那种似乎与近代西方公民社会最接轨的“罗马法”其实是在拜占廷时代才最后定形的。而在此之前,古罗马的大部分历史中都以极为发达的父权制大家族闻名。我们曾提到秦时(西汉其实也如此)宗族关系极度淡漠的情况,而就在与秦汉大致同时,从共和国到帝制罗马的前、中期,罗马法都把父权与夫权置于重要地位。那时罗马私法规定的各种民事权利大都只对父家长而言,包括最重要的“物权”(财产权)在内。罗马社会极重家族神、家族祭祀与家族谱系,所谓公民权那时实际上就是“有公民资格的父家长权”,甚至连公民中最底层的“无产者”也不例外——“无产者”即古拉丁语 proletarius,原义即“只有家族”,谓除此而外别无所有也。罗马氏族组织与氏族长老(即所谓贵族)在共和时代的政治中起着重要作用。而到帝国时代虽然氏族关系已淡化,但涵盖数代人的家族组织仍是很重要的。与承认父子异财、夫妻异产的秦律形成鲜明对照的是:罗马法直到帝制时代一直认为家长对子弟的权利等值于奴隶主对奴隶的权利,并把子女与奴隶及其他家资一样视为家长的财产。但正是在这样的条件下,罗马形成了在那个时代的世界上最发达的古典公民社会。如所周知,近代公民社会的许多权利规范都是从它起源的。

只是到帝国晚期,罗马父权与家长制家族的法律地位才趋于崩溃。君士坦丁大帝时期的家庭与婚姻法改革使无夫权婚姻基本取代了有夫权婚姻,并使家属逐渐摆脱家长的控制而取得自权人的地位(Gruhbos 1995)。民法权利,包括财产权的主体,也渐从家长泛及于每一自由人个体。到了拜占廷时代,宗族纽带已经解体到这种程度:甚至连包含家族名称的拉丁式姓名也已被废弃,而在 8 世纪前后为不含家族名的希腊式姓名逐渐取代了(Treadgold 1997, 394–5)。无怪乎经查士丁尼整理后的“罗马法”“现代化”到了如此程度,以致如马克思所说:个人本位的近代市民社会甚至用不着怎么修改便可把它作为“经典性的法律”来使用。

然而耐人寻味的是:这种家族共同体的解体与家(族)长权的崩溃在拜占廷并没有导致公民权利的发展。相反,拜占廷社会走上了“东方化”的老大帝国之路,在政教(东正教)合一的专制极权之下,把罗马公民社会的古典基础完全消解了。这便引起了当代罗马法史研究中的有趣的讨论。有人认为:“罗马严格的个人主义在后古典时代(按即帝国晚期及拜占廷时代)屈服于一种更偏重社会利益的评价,并在这方面出现了许多对所有权的限制。”有人却指出,后古典法与查士丁尼法“可能恰恰表现为一种对个人主义的确认”。类似地,有人认为对家属的宽待等等体现了“新时代的基督教人道精神”,有人却发现在拜占廷化过程中随着公民身份含义的蜕变,即“人格的意义在降低”(格罗索 1994,435)。其实,这里的关键在于拜占廷的宗族小共同体纽带与家长权并不是(如近代那样)由公民契约纽带与公民个人权利来冲垮的,而是由从戴克里先到查士丁尼的专制国家大共同体桎梏与东方式皇权来摧毁的。消除了“宗法性”的拜占廷式罗马法尽管在技术上(成文法的形式结构上)“先进”得很,以至近代法律几乎可以照搬,然而拜占廷的立法精神却比古典罗马距离近代法治更为遥远。正如牛津大学拜占廷学大师奥勃连斯基所云:近代法治的基础是公民权利本位,而拜占廷法的基础是“广泛的国家保护”;近代法治的本质是法的统治(the rule of law),而拜占廷法的本质则是“君主本人根据他颁布之法律进行统治”(ruled by a sovereign himself subject to the laws he has promulgated)(Oborensky 1971, 317–20)。这样的“反宗法”,与其说是提高了家属的人格,不如说是压低了家长的人格,与其说是使家属成为了公民,不如说是使家长从公民沦为了臣民。无疑,那种全能的、至上的、不容任何自发组织形式存在的“大共同体”对公民个性的压抑,比“小共同体”更为严重。在罗马时代,真正享有充分公民权利的只是少数人(自由公民中的父家长),但至少对这一部分人而言他们的个人权利、人格尊严与行为能力是受到尊重的,在此基础上就可以通过契约整合而产生自治的公民社区和更大的公民社会。而拜占廷帝国那全能的“大共同体”则“平等地”剥夺了一切人的公民权利,它不仅抑制了“小共同体”的发展,更压抑了人的个性发展。

无怪乎在罗马法一度湮灭的西部“蛮族国家”后来会发生“从宗族社会到公民社会”的演进(并且在这一演进中产生了以公民权利的“复兴”为基础的“罗马法复兴”),而在专制皇权下发展了如此完善的“民法大全”的拜占廷反而走上了老大帝国的不归路!

华夏文明与罗马文明在“文化”上差异极大,但在大共同体本位的趋势下发展出一种“反宗法的臣民(非公民)社会”,却是秦汉与拜占廷都有类似之处的。与拜占廷民法的非宗法化或“伪现代化”相似,秦汉以来中国臣民的“伪个人主义化”也十分突出。尽管近年来的人类学、社会学家十分注意从社区民俗符号与民间仪式的象征系统中发现村落、家族的凝聚力,但在比较的尺度上,我十分怀疑传统中国人无论对血缘还是地缘小群体的认同力度。且不说以血缘共同体而论秦汉法家传统下的“五口之家”不会比罗马父权制大家族更富于家族主义,以地缘共同体而论近代中国小农不会比俄国米尔成员更富于村社意识,就是在无论村社还是宗族都远谈不上发达的前近代英国,那里的“小共同体意识”也是我们往往难于理解的。从中学到大学,英国历史上的“圈地运动”往往都被我们的教师讲解为、也被学生理解为“跑马占圈”式的恶霸行径。及至知道那其实是突破当时的村社习惯而实行“自由”择佃(赶走原来的佃户而把土地租给能出更高租金的外来牧羊业者),则往往会大惑不解:这算什么事?咱中国自古不就如此的吗?不仅把土地出租给外村人,就是卖给了外村人,在传统中国农村不也司空见惯么?何以英国“地主”只是把土地租给(还不是卖给)外村人便会引起如此强烈的社会反应?而自古以来就如此“开通”的中国人怎么就始终弄不出个“资本主义”呢?我们在唐诗中就可以读到诸如“客行野田间/比屋皆闭户/借问屋中人/尽去作商贾”(姚合:《庄居野行》)这样的情景,除了官府经常搞“检籍”“比户”这类户口控制外,社区几乎是不管的。而在许多前近代的欧洲国家,即使是非农奴的“自由农村”,小共同体的控制力也很强。不要说“尽去作商贾”,就是搬到村外去盖个房子也要突破村社习惯的阻碍。像俄罗斯一直到十月革命时,自由散居的“独立农户”仍然是一种阻力重重之下的新生事物。(金雁 等 1996)。

无疑,与其他前近代文明相比,中国人(中国“小农”)对社区(而不是对国家)而言的“自由”是极为可观的。然而中国人(中国“编氓”)对国家(不是对社区)的隶属就更为可观。如今在社会学界有人引西人之论,说中国也如西欧一样,“民族国家”只是一种“现代性”的产物,而在经济史界又有论者把中国传统经济研究分为三派:笔者被列为“权力经济”论者,美国学者赵冈等被列为“市场经济”或自由经济论者,而国内经济史界的主流则被列为似乎是居于二者之间的“封建地主经济”论——这种经济似乎既没有赵冈等人说的那么自由,又没有笔者说的那么带有强权性质。然而实际上,该论者所说的那两种“极端”之论是可以统一的,而且都比那种主流的“中庸”之论近于事实。就小共同体范围而言,中国的“小农”的确比外国的村社社员“自由”。哪个村社能允许传统中国这样的自由租佃、自由经商?而就大共同体尺度看,中国的“编氓”又的确比外国的“前国家”居民更受制于强权。哪个“前国家”能像传统中国那样逼得国民一次次走投无路而形成周期性的社会爆炸?但说起来,大共同体本位的趋势并非中国传统独有。古罗马向拜占廷的发展亦然。就家(族)内而言,拜占廷的家(族)成员比古典罗马更“自由”;就国民而言,拜占廷臣民却比罗马公民更受奴役,只不过这一趋势在古代中国要更突出得多了。

\section*{里—社—单合一:传统帝国乡村控制的一个制度性案例}

中华传统帝国的农村基层组织是怎样的? 应当说这是个前沿性的探索领域,两千年来这种组织的沿革一本书都未必能说清,不过可以肯定它决不像那些“伦理自治”“宗族本位”“古代自由主义的小政府大社会”之类说法那么简单。我们可以以汉代的里—社—单体制作为一个制度性案例略作剖析。

汉代的农村最基层组织过去人们提得较多的是正史中乡—亭—里体系中的“里”制。近年来人们根据出土资料与文献对勘,又对与“里”平行的“社”“单”之制有了较多的认识。尤其是俞伟超先生以汉印、封泥、碑碣结合文献作出的单(僤、弹)制考证(俞伟超 1988)意义重大,引起了广泛关注。俞先生认为单(僤、弹)是“中国古代的农村公社组织”,而台湾学者杜正胜先生则称之为“农作协助团体”,是“各种不同性质的结社”(杜正胜 1990)。按俞说之“农村公社”概念系来自马克思理论,尤其是马克思关于古代东方专制国家以农村公社为基础的说法,它与当今国际史学界主流多把米尔、马尔克这类村社组织看作乡土自发的小共同体而区别于国家基层组织的观念不同,按后一观念,“单”是基层组织,不能算村社的。

据现有资料,里、社、单都是同级同范围并往往同名的基层设置,常常并称为“里社”,“社弹”“里单”等,从“宜世里”“宜世单”,“侍廷里僤”“众人社弹”等称呼看,当时一里必相应设有一社一单。

里为行政组织,设有里唯(里魁、里正)、里父老、里佐、里治中等职;社为祭祀组织,是当时的“意识形态系统”,设有社宰等职;单为民政、社会组织,功能最复杂,设职也最多:出土官印就有“祭酒(祭尊)”,是为单首;“长史”“卿”,均为单副;“三老”(敬老,父老)掌教化;“尉”掌“百众”(民兵);“平政”掌税役;“谷史”掌单仓(又有谷左史、谷右史之分);“司平”掌买卖;“监”“平”(又有左平、右平)掌讼、狱;“厨护”(又有左厨护、右厨护)掌社供;“集”(又有左集、右集)掌薪樵;“从”掌簿书;等等。

以上诸职皆有出土官印为证。汉之一里为户仅数十,而以上三系统设职就不下20个。虽未必每里全设,亦足惊人。以上诸职连同承担情治、信息职能的亭邮系统,上接乡一级诸机构,组成了一个严密的控制网络,如下表:
\begin{enumerate}
\item 行政系统
  \begin{figure}[H]
    \centering
    \begin{tikzpicture}
      \begin{scope}[xshift=-1.4cm]
        \ifafourpaper
        \draw (-1.2,1.5) -- (-1.2,-1.5);
        \node[anchor=east] at (-1.2,0) {县};
        \draw (-.6,0) -- (-1.2,0);
        \else
        \draw (-1,1.5) -- (-1,-1.5);
        \node[anchor=east] at (-1,0) {县};
        \draw (-.5,0) -- (-1,0);
        \fi
        \draw (0,1.5) -- (0,-1.5);
        \draw (0,0) node[anchor=east] {乡};
        \node[anchor=west] at (0,1.25) {有秩};
        \node[anchor=west] at (0,.75) {(啬夫)};
        \node[anchor=west] at (0,.25) {三老};
        \node[anchor=west] at (0,-.25) {游徼};
        \node[anchor=west] at (0,-.75) {乡佐};
        \node[anchor=west] at (0,-1.25) {“乡廷步吏”};
      \end{scope}
      \begin{scope}[xshift={\ifafourpaper2.2cm\else1.8cm\fi}]
        \ifafourpaper
        \draw (-1.2,1.5) -- (-1.2,-1.5);
        \draw (-.66,0) -- (-1.2,0);
        \else
        \draw (-1,1.5) -- (-1,-1.5);
        \draw (-.55,0) -- (-1,0);
        \fi
        \draw (0,1.5) -- (0,-1.5);
        \draw (0,0) node[anchor=east] {里};
        \node[anchor=west] at (0,.75) {里唯(魁、正)};
        \node[anchor=west] at (0,.25) {里父老(三老)};
        \node[anchor=west] at (0,-.25) {里佐};
        \node[anchor=west] at (0,-.75) {里治中};
      \end{scope}
    \end{tikzpicture}
  \end{figure}
\item 情治、信息系统
  \begin{figure}[H]
    \centering
    \begin{tikzpicture}
      \ifafourpaper
      \draw (-2.05,1.5) -- (-2.05,-1.5);
      \node[anchor=east] at (-2.05,0) {县尉};
      \draw (-1.42,0) -- (-2.05,0);
      \else
      \draw (-1.9,1.5) -- (-1.9,-1.5);
      \node[anchor=east] at (-1.9,0) {县尉};
      \draw (-1.25,0) -- (-1.9,0);
      \fi
      \draw (0,1.5) -- (0,-1.5);
      \draw (0,0) node[anchor=east] {乡游徼};
      \draw (1.2,0) node[anchor=east] {亭};
      \draw ({\ifafourpaper.55\else.65\fi},0) -- (0,0);
      \draw (1.2,1.5) -- (1.2,-1.5);
      \node[anchor=west] at (1.2,1) {亭长};
      \node[anchor=west] at (1.2,.5) {亭侯};
      \node[anchor=west] at (1.2,.0) {亭佐};
      \node[anchor=west] at (1.2,-.5) {求盗};
      \node[anchor=west] at (1.2,-1) {“亭部吏卒”};
      \ifafourpaper
      \draw (3.6,1.5) -- (3.6,-1.5);
      \draw (3.6,0) -- (4.05,0);
      \node at (4.3,0) {邮};
      \else
      \draw (3.35,1.5) -- (3.35,-1.5);
      \draw (3.35,0) -- (3.8,0);
      \node at (4,0) {邮};
      \fi
    \end{tikzpicture}
  \end{figure}
\item 意识形态系统
  \begin{figure}[H]
    \centering
    \begin{tikzpicture}
      \ifafourpaper
      \draw (-1.9,.8) -- (-1.9,-.8);
      \node[anchor=east,align=center] at (-2.58,0) {“公社” \\ (乡社)};
      \draw (-1.9,0) -- (-2.58,0);
      \else
      \draw (-1.65,.8) -- (-1.65,-.8);
      \node[anchor=east,align=center] at (-2.15,0) {“公社” \\ (乡社)};
      \draw (-1.65,0) -- (-2.15,0);
      \fi
      \node[align=center] at (0,0) {“置社” \\ (里社、社弹、书社)};
      \ifafourpaper
      \draw (1.9,.8) -- (1.9,-.8);
      \node[anchor=west] at (1.9,0) {社宰};
      \else
      \draw (1.65,.8) -- (1.65,-.8);
      \node[anchor=west] at (1.65,0) {社宰};
      \fi
    \end{tikzpicture}
  \end{figure}
\item 民政、社会系统
  \begin{figure}[H]
    \centering
    \begin{tikzpicture}
      \node[align=center] at (0,0) {里单(僤、弹)};
      \ifafourpaper
      \draw (-1.32,0) -- (-2.16,0);
      \draw (-2.16,3) -- (-2.16,-3) (1.44,3) -- (1.44,-3);
      \node[anchor=east] at (-2.16,0) {乡};
      \node[anchor=west] at (1.44,2.75) {祭酒(尊)};
      \node[anchor=west] at (1.44,2.25) {三老——左、右父老};
      \node[anchor=west] at (1.44,1.75) {长史};
      \node[anchor=west] at (1.44,1.25) {卿};
      \node[anchor=west] at (1.44,.75) {尉——“百众”};
      \node[anchor=west] at (1.44,.25) {平政};
      \node[anchor=west] at (1.44,-.25) {谷史——左史、右史};
      \node[anchor=west] at (1.44,-.75) {司平};
      \node[anchor=west] at (1.44,-1.25) {监(平)——左平、右平};
      \node[anchor=west] at (1.44,-1.75) {厨护——左厨护、右厨护};
      \node[anchor=west] at (1.44,-2.25) {集——左集、右集};
      \node[anchor=west] at (1.44,-2.75) {从(治中从事)};
      \else
      \draw (-1.1,0) -- (-1.8,0);
      \draw (-1.8,2.5) -- (-1.8,-2.5) (1.2,2.5) -- (1.2,-2.5);
      \node[anchor=east] at (-1.8,0) {乡};
      \node[anchor=west] at (1.2,2.31) {祭酒(尊)};
      \node[anchor=west] at (1.2,1.89) {三老——左、右父老};
      \node[anchor=west] at (1.2,1.47) {长史};
      \node[anchor=west] at (1.2,1.05) {卿};
      \node[anchor=west] at (1.2,.63) {尉——“百众”};
      \node[anchor=west] at (1.2,.21) {平政};
      \node[anchor=west] at (1.2,-.21) {谷史——左史、右史};
      \node[anchor=west] at (1.2,-.63) {司平};
      \node[anchor=west] at (1.2,-1.05) {监(平)——左平、右平};
      \node[anchor=west] at (1.2,-1.47) {厨护——左厨护、右厨护};
      \node[anchor=west] at (1.2,-1.89) {集——左集、右集};
      \node[anchor=west] at (1.2,-2.31) {从(治中从事)};
      \fi
    \end{tikzpicture}
  \end{figure}
\end{enumerate}

如此复杂的基层组织,即便在今天也难想像。可以设想“制度”与实际是有差距的,但即便差距再大,也与今人所理想的“伦理自治”不可同日而语。重要的是:即便实际设置没有这般复杂,它的性质是清楚的,即它是一种国家组织的下延,而不是自生自发的草根组织。

这由以下数点可知:
\begin{enumerate}
\item 它是政教合一(里社、单社合一)、政社合一(里单合一)的一元化体系,并具有行政主导的特点。正如出土的《侍廷里父老僤买田约束石券》所示:当时立僤的主持人是里官(里治中),可见在里—社—单体制中,里是主干,而社、单都是附着在里上的。而里的本质不是别的,正是法家运动为之奠定基础的专制国家对编户齐民的直接(即不经村社,宗族等中介)管制,即所谓“闾里什伍”之制。

\item 它的合法性是自上而下的,所谓“给事县”,所谓“里正比庶人之在官”。而“庶人之在官”即为“吏”,《汉书・尹赏传》所谓“乡里少吏”,“乡吏、亭长、里正、父老、伍人”皆属之。它的择人标准,据史载有“强谨”“訾次”“德望”“年长”等项.所谓强谨,即能办事(强)承上意(谨)即可为吏,而不必求民间的道德形象,像刘邦这样乡里视为“无赖”的人决谈不上德高望重,却可以当亭长,就是典型之例,“訾次”就是论财力,以便能应付职役。显然这两项标准都是从国家而不是从社区考虑的。至于“德望”与“年长”的确有些“伦理自治”的味道,但其权威也必须由上面来确认的。

\item 它在形式上也摹拟官场。里印、社印、单印都按当时所谓“方寸官印之制”刻成,而“祭酒”“三老”“尉”“治中”“长史”“卿”等称谓也是在上级官场有相应设置的。这样的组织显然不是“民间结社”而是“基层政权”。

\item 它是一整套非宗族的政治设置。与秦汉(主要指西汉)时实行的强制分异、“不许族居”、“父子兄弟同室共息者为禁”的气氛相应, 里—社—单组织都没有什么族缘色彩。迄今所知的里、社、单名多是“吉语”(如宜世、奉礼、常乐等)或方位(如亭南、中治等),从无后世之“李家庄张家寨”之类族姓称谓。汉以后出现的“村”初亦如此。正如宫崎市定所言,“村”即“邨”,起源于屯田,它也是按国家安排设置的。而存世《侍廷里单石券》题名共25个“父老”,这是该里(单)的“领导班子”,25人中至少有6个姓氏,显然并无宗族背景。
\end{enumerate}

这种基层组织靠什么养活?汉代乡级组织包括亭在内,基本是政府财政支持的,其中不少大约直接取给于当地上交的政府税收。出土的西汉江陵市阳里、郑里与当利里《算钱(人头税)录》都有不少把收上的部分“算钱”上交乡里转为“吏奉(俸)”的记录。至于里级组织则主要是自筹费用,包括征收的“社钱”和《侍廷里单石券》记载的“敛钱”购置的“容田”收入等。但不能说,只要不是政府直接开支养活的组织就是“自治”组织,关键在于其赖以筹资的权威资源何来。由前述可知这一资源也主要来自上面。这样的组织有多少“自治”色彩是可疑的。

总之,秦汉时代我国传统帝国的农村基层控制已相当发达和严密。汉以后除东汉后期到北魏的宗主督护制时期帝国根基不稳外,也一直维持着专制国家对“编户齐民”的控驭。

而基层以上在地方与中央的关系中集权的趋势就更明显。近年来曾有人极言“国家能力”问题,他们断言:我国历代王朝的崩溃,都是由于“国家财政、尤其是中央财政汲取能力下降”的结果。这真不知从何谈起。实际上除少数例外(如东汉末)外,我国多数王朝的崩溃都恰恰是在官府尤其是朝廷的“汲取能力”极度亢进而使民间不堪忍受之时。秦末大乱正是朝廷集中全国大部分人力物力滥兴营造的结果,隋末、元末的大乱也有类似背景。西汉末(新莽)厉行“五均六管”等“汲取”之政,新莽灭亡时,全国的黄金仅集中在王莽宫中的库藏就达70万斤之巨,其数量据说恰与当时西方整个罗马帝国的黄金拥有量相当!而明亡之时按黄宗羲的说法,则是全国“郡县之赋,郡县食之不能十之一,其解运至于京师者十之九”(《明夷待访录・田制一》)。看看当时各地方志《赋役志》中有关“存留、起解”的记载就会明白,黄氏所说并非虚语。试问当今天下有几个国家“中央财政所占比重”能达到如此程度?当然这只是就法定税赋而言,当中央把郡县的法定收入几乎尽数起解之后,地方政府的开支只能多依赖杂派,而基层亦复如此。“明税轻,暗税重,横征杂派无底洞”这样一种痼疾在我国历史上是古已有之。而基层控制也就成了这种痼疾之前提。直到痼疾引发社会爆炸,基层也就失控了。

\currentpdfbookmark{近古宗族之谜}{lineage}
\section*{近古宗族之谜}

总之,与前近代西方相比,中华传统帝国的统治秩序具有鲜明的“国家(王朝)主义”而不是“家族主义”特征。如果说中古欧洲是宗族社会(lineage society)的话,古代中国则是个“编户齐民”社会。在历史上,郡县立而宗法“封建”废,“三长”兴而“宗主督护”亡,这类事情与礼教对大家族的褒奖构成了奇特的互补。当然,在“儒表法里”体制下统治者宣扬宗法礼教并不完全是为了骗人而自己根本不信。但礼教的真正意义在于反“个人主义”而不在于反“国家主义”。专制国家对宗族组织的支持是为了抑制臣民个体权利,而不是想扩张“族权”,更不是支持宗族自治。朝廷对宗族文化的赞赏是为了压抑臣民的个性,而不是真要培养族群的自我认同。因此我们切不可对统治者提倡“大家族主义”的言词过于当真。明初,“浦江郑氏九世同居,明太祖常称之。马皇后从旁惎之曰:以此众叛,何事不成?上惧然,因招其家长至,将以事诛之”(方孝标:《钝斋文选》卷六)。这个故事是极有典型意义的。在清代,乾隆年间的江西官府曾经有“毁祠追谱”之举,以图压制民间宗族势力。而广东巡抚还提出由朝廷对大族的族产实行强制私有化,以削弱宗族公社的经济力量。乾隆不仅对此表示同意,还旨令在全国各地推行,要求对那些“自恃祠产丰厚”而尾大不掉的强宗巨族进行打击,把各该族祠产业清查后分给族人(《(光绪)钦定大清会典事例》卷399《礼部・风教》)。乾隆帝还在给《四库全书》馆臣的“圣谕”中明确规定:“民间无用之族谱”不得收入《全书》(《四库全书总目》卷首,《圣谕》)。

事实上,历代统治者不管口头上怎么讲,实际对“法、术、势”的重视远远超过四维八德。而法家传统是极端反宗族的,它强调以专制国家本位消除家族本位,建立不经任何阻隔而直达于每个国民个人的君主极权统治。它主张以皇权(以及完全附属于皇权的吏权)彻底剥夺每个国民的个人权利,并且绝不允许家族、村社或领主截留这些权利而形成隔在皇权与国民个人之间的自治社区。换句话说,它不仅容不得公民个人权利,也容不得小共同体的权利。儒家的“家国一体”在这种情况下只对于皇帝一家是真实的:它只意味着皇帝“万世一系”的家天下,而不支持别的“世系”存在。某个家长“提三尺剑,化家为国”([前蜀]王建:《诫子元膺文》)而建立起自己的皇权后,就决不允许别人起而效尤,膨胀“家族”权利。另一方面,历史上许多所谓的“农民起义”,从西晋的乞活直到清代的捻子,也都有宗族豪酋聚族抗官的背景。抗官成功,“提三尺剑,化家为国”的故事便又一次上演,但同时对民间宗族势力的戒备,也成为这些昔日宗族豪酋操心的事。

到了近代,更出现了受“西化”影响以“家族自治”为旗号而反抗传统国家专制的改革派“新潮宗族”活动。这是以往人们很少注意的。如清末以走“英国之路”为标帜的立宪派地方自治运动中就出现了“家族自治”的呼声。当时重要的南方立宪派团体广东地方自治研究社有38个集体成员为支社,其中就包括五个家族自治研究社(所)(候宜杰 1993,140)。这显然已经超越了传统宗族豪酋抗官的窠臼,而赋予了小共同体维护权利的努力以新的意义。

综上所述,近古—近代中国传统社会中自然形成的小共同体(宗族是其最常见的形态)是一种十分复杂的现象,它可能既不像人们所认为的那样发达(由于大共同体本位的制约),也不像人们所认为的那样“传统”。但是,近代以来中国人和外国人,中国的从最激进者到最保守者的各派力量,却大都认为中国的宗族是既发达而又“传统”的。区别只在于他们有的反对、有的则维护这种“传统”。乃至今日,人们仍把对宗族伦理、血缘纽带与家长制的依恋当成“中国传统”与民族性的基础,虽然他们有的骂它是“劣根性”,有的赞之为民族魂。

然而中国人真的天生就比其他民族更恋“宗族”吗?现代化即以经济市场化和政治民主化为表征的个性化过程对传统共同体纽带的消解在“中国文化”中就不起作用吗?其实有许多材料表明:即使在传统时代,中国人的宗法观念也并不比其他民族的更耐商品经济的“侵蚀”。明清时代兴旺的宗族文化中就有不少人惊呼市井的威胁。如明代名宦广东人庞尚鹏在他那部《庞氏家训》(近古宗族法规类著作中极有名有一部)里写道:庞氏族人应当远离市井繁华的广州城,否则“住省城三年后,不知有农桑;十年后,不知有宗族。骄奢游惰,习俗移人,鲜有能自拔者。”这类劝诫表明,在“宗族文化复兴”的明清东南地区,市场经济对宗法关系的冲击同样明显,以至卷入其中者不久便“不知有宗族”,而且这不仅是个别现象,而是“习俗移人,鲜能自拔”。看来仅有“儒家文化”并不足以使宗法关系具有抵御市场侵蚀的“免疫力”。

然而近古以来宗族组织的确有“逆逻辑发展”之势,即市场关系发达的东南沿海宗族盛于江南,江南又盛于华北、内地,清盛于明,明又盛于宋元等,这是为什么?而近代以来人们都有中国宗族既发达又“传统”的看法,这又是为何?

后一个问题现在看来是可以理解的。近代中西“文化碰撞”之时西方已经完成了“从宗族社会到公民社会”的演进,相形之下“宗族社会”便显得成了中国的专有特征了。而近代以来中国的启蒙、现代化与激进思潮又是在救国救亡的民族危机背景下发生,人们痛感国势孱弱、国家涣散,在强国梦中很难产生对大共同体本位的“国家主义传统”的深刻反思,个性解放与个人权利的近代意识主要是冲着小共同体桎梏即“宗族主义”的束缚而来,便成为理所宜然。从严复、梁启超到孙中山都在抨击宗族之弊的同时发展着某种国家主义倾向,尽管这种国家主义所诉求的是现代民族国家而非传统王朝国家,但它毕竟会冲淡对“大共同体本位”之弊的反思。在此潮流中的五四新文化运动也是反“儒”而不反“法”,在对宗法礼教发动激进抨击的同时并未对儒表法里的传统作认真的清理,个性解放的新文化在反对宗族主义的旗号下走向了国家主义,后来的中国式马克思主义实际上是把这一倾向推到了极端——后来在文革中“马克思主义”者发动的“批儒崇法”、反孔扬秦(始皇)运动便是这一逻辑的结果。

当然,所谓反宗族主义不反国家主义并不是说那时的人们只反族长不反皇帝,中国人那时对皇权专制的批判不亚于对宗族桎梏。然而这种批判的主流只是把传统专制当作皇帝个人的或皇帝家族的“家天下”来反,而缺乏对大共同体扼杀公民个人权利(甚至也扼杀小共同体权利)的批判。似乎只要不是一姓之国而是“人民”之国,就有理由侵犯乃至剥夺公民个人自由。皇帝专制是恶而“人民”专制是善的观念就此流行。

除了爱国救亡取向的影响外,中国人接受的西学中存在的“问题错位”也是重要原因。西方的近代化启蒙与西方个性解放思潮都是针对他们那小共同体本位的中世纪传统而来,而国家主义在他们那里是一种近代思潮——正如民族国家在他们那里是近现代现象一样。尤其是在近代化中实现民族统一的德国及出现过“人民专制”的法国,左、右两种国家主义都很流行。偏偏两次大战之间的欧洲又是个“国家主义的黄金时代”,那时输入中国的种种国家主义思潮,更进一步加剧了只反宗族主义不反国家主义的倾向。所有这些因素的综合结果,便造成了中国传统的批判者与捍卫者都把目光盯着宗族主义的现象,造成了“中国传统社会是家族本位社会”的误解。

至于近古以来宗族组织在中国的“逆逻辑发展”则是一个复杂的现象。首先,近古出现的许多宗族是地方官僚甚至官府出面组织的“官办宗族”,它本身就是大共同体本位的产物而不是什么“伦理自治体”。这方面的典型是关中,关中与东南沿海相比,宋元以来一直是个宗族组织极不发达的地方,聚落多为杂姓,族田族产罕闻,修谱之风不兴,但却有秦政商君之法“闾里什伍之制”的遗风,里甲基层组织甚为发达。近代合阳等地实行“地丁属地,差徭属人”之法,由里甲组织负责调派差徭。随着丁夫之征日剧,农民不胜其扰,里甲效率下降,这时地方官府便发动大姓出面,实行里甲宗族化以强化“差徭属人”的组织。当时通过政府行为把一姓之人按里甲划分宗支,或在族姓大分散小聚居的地方调整里甲、按姓设甲。这样划分的宗支实际上并不符合自然血缘谱系与儒家经典的宗法理论,地方官也承认此种宗族不伦不类,但却符合官方需要:“属人之中,宗法寓焉。故一问某大族之里甲,即生长分、次分,与同姓不同宗者之关系。或分甲分里分乡,而次序依然不紊,亦不可谓毫无优点也。”(范清丞:《合阳赋役沿革略》)

更多的宗族“官办”色彩没有这么鲜明,但地方官府和乡绅在宗族组织化中的作用同样很突出。值得指出的是以科举制为前提而形成的“乡绅”这一阶层,他们被今人称为“地方精英”,他们因为张口儒学闭口礼教而被人视为“宗族本位”的代表。实际如前文所提到的,科举作为一种制度本身就是直接否定察举、士庶、门阀之制的宗族色彩的。士子们打破了宗族身份界限而完全以个人身份接受专制国家的“智力测验”,由此被网罗入国家机器。他们在外任官时完全是食君禄理君事的“国家雇员”,在家乡也由政府(而不是由社区或宗族)的优免政策保障经济利益与政治权势,其“权威资源”完全是自上而下的,其主要角色也只能是“国家经纪”而决非“保护型经纪”。这一点在明末大乱中表现的至为明显:当时地方上那些守土自保、既抗拒“流寇”也抵制官府的“土寨”几乎都不是由乡绅、而是由无功名无缙绅身份的平民富户主持的(秦晖 1986)。当然,在籍乡绅——未入仕的候补官员和致仕返乡的前官员——与任职当地的外籍“朝廷命官”相比,能够多考虑一些地方利益与乡土关系,但他们与科举制以前的士族、宗主,与国外的领主、村社首领相比,那点“地方性”就太不足道了。同时,“地方利益”也不等于宗族利益。由于科举本身是非宗法的,科举出身者未必在宗族中居优势地位(如长房、嫡派等),由他们组织的宗族虽未必像关中的“里甲化宗族”那样不伦不类,也难免有违自然血缘谱系,行政考虑高于“伦理考虑”,这离血缘小共同体自治就更远了。

因此,宋元以来的宗族兴盛未必与地方自治有关,更未必与大共同体本位的传统相悖。勿宁说,由科举出身者(更不用说由当地官府)控制“宗族”之举本身就是“儒表法里”的一种形式,是大共同体本位的一种表现。

但明清以来也有另一种情况,即因大共同体本位的动摇与小共同体权利的上升而导致的宗族现象。清代沿海地区随市场经济的发展而反趋兴盛的宗族,便带有这种色彩。如广东珠江三角洲盛行沙田农业,沙田为冲积滩地经人工围垦而成,面积随冲积而增减,地界难以划定,经常引起争夺。当大共同体本位体制稳定的明前期,明地方官府是沙田开发的主导,在组织围垦、平息纷争中起关键作用。但明中叶起沙田开发开始向民间主导转化。清乾隆时发明石围技术,民间投资大增,一些有势力的大姓组织族人合股开发,使宗族势力膨胀起来,逐步排挤了官府的影响。清同治后朝廷财政危机,在广东出售屯田,宗族势力因而控制了整块沙坦,规划大围,到光绪时出现了具有浓厚商业因素的围馆与包佃,成为筑围的投资方。

然而有趣的是随着官府控制的削弱与民间商业性沙田开发的发展,“宗族公社”在珠江三角洲膨胀起来,由于沙田多为宗族公产,许多县的耕地中族产多已占一半以上,甚至高达80\%。官府忌于宗族势力,乾隆时一度企图强行推行族产私有化,但并无成效。到清末珠三角农村几成宗族的天下,“有时这些组织是建设地方之领导,有时则是对抗官府的主要分子”(黄永豪 1987)。而清末立宪派的“家族自治”运动就是在这种背景下发生的。

这样的家族(宗族)在微观上或许与传统的宗法共同体并无不同,而与近代的契约性公民社团绝不相类。但在宏观上它作为大共同体本位的瓦解力量却可能具有新的意义。正如拜占廷罗马法(或秦汉法家体制)中的家庭关系微观上看十分“现代化”,但在宏观上它作为大共同体本位的构成力量却比罗马式大家族距离公民社会更远。近古—近代中国农村宗族与其他小共同体的“逆逻辑发展”是否具有这种因素呢?若如庞尚鹏所言,广东人并非特殊的“宗族迷恋者”而近代的“家族自治”又确为官府所不喜,我们就不能排除这种可能。

\currentpdfbookmark{公民与小共同体的联盟?}{alliance}
\section*{公民与小共同体的联盟?}

传统社会的反近代化机制无疑有儒家色彩的一面,即大共同体与小共同体都抑制个性,父权制家族桎梏与专制国家桎梏都阻碍着自由交换、竞争与市场关系的发展,阻碍着民主、人权与公民社会的形成。但这种反近代化机制更确有非儒家色彩(或曰法家色彩)的一面,即“大共同体”不仅抑制个性,而且抑制小共同体,不仅压抑着市场导向的个人进取精神,而且压抑了市场导向的集体进取精神。近古中国政治中枢所在的北方地区宗族关系远不如南方尤其东南一带发达,但公民社会的发育却比南方更为艰难,这无疑是重要原因之一。

在前近代社会中,束缚个性发展的共同体桎梏是多种多样的,而个性发展的进程往往不可能一下子同时摆脱所有的共同体桎梏而一步跨入“自由”状态。因此,个性发展的一定阶段就可能表现为桎梏性较小的共同体权利扩张,对主要的共同体桎梏形成制衡与消解机制。如所周知,中世纪欧洲城市中的行会是市场关系发达的障碍。但在早期正是行会在与领主权作斗争、争取城市自治的进程中发挥了重要作用。后来出现的“公民与王权的联盟”更是如此。在缺少中央集权专制政体的西方,“民族国家”形成很晚,“大共同体”长期处于不活跃状态,阻碍人的个性发展的主要是采邑、村社、行会、宗族等小共同体的束缚。在这种情况下,大共同体的权力对于冲破小共同体桎梏、从而解放个性,是有着积极作用的。因而公民(市民)可以与王权携起手来反对领主权与村社陈规,而在依附型小共同体的废墟上建立起公民社会的基础:公司、协会、社团、自治社区等。随着这些基础的建立,公民权利成长起来后,才转而向王权及其所代表的“大共同体”发起挑战,用民主宪政的公民国家取代“王朝国家”。

而在传统中国不可能出现这种情形。相反,由于传统中国“大共同体”的桎梏比“小共同体”强得多,因此如果说在西方王权虽从本质上讲并非公民社会的因素,但它在一定的发展阶段上可以有助于市民社会成长,那么在中国,包括血缘、地缘组织(宗族、村社等)在内的小共同体,即便内部结构仍很“传统”,但只要它对大共同体本位体制而言具有自治性,则它在一定阶段上也可能成为推动市场关系与人的个性发展的有利因素。换句话说,如果在西方,从小共同体本位的传统社会向个体本位的公民社会演进需要经过一个“市民与王权的联盟”(本质上即公民个人权利与大共同体权力的联盟)的话,那么在中国,从大共同体本位的传统社会向公民社会的演进可能要以“公民与小共同体的联盟”作为中介。 所谓“联盟”当然是个象征性说法,不一定指有形的盟约(西方历史上的“市民—王权联盟”也并非实指),而是意味着争取公民个人权利与争取小共同体权利二者间形成客观上的良性互动。

在社区自治与自治性社区权利极不发达的传统中国,与市场关系的发展相联系的小共同体(包括家族组织)可以在某种程度上起到社区自治的功能,并以其集体进取精神克服大共同体的束缚,从而为个性的发展打开突破口。明清时期东南地区宗族关系与商品经济同步而“逆逻辑发展”的事实表明了这种可能性,但它最终能否像西方王权一度有助于市民社会的建立一样,为中国打开一条新路,则历史并未作出答案,因为近代以后中国原来的发展轨迹中断了。不过改革时期东南地区再度出现宗族共同体与市场同步繁荣的局面,却十分耐人寻味。

改革后东南农村出现的“乡镇企业”被世人目为“奇迹”。而乡企的发展史初看起来似五花八门,典型的如“温州型”乡企多是私有制,而“苏南型”乡企在1996年大转制前则是“集体企业”为主。于是关于“乡企奇迹”的原因也就有了”市场动力”与“新集体主义”等彼此抵牾之说。然而人们却很少注意到,这些或公或私的企业分布却有一个共同点,即它们主要是在“市场网络所及、国家控制弱区”发展起来的。创造出温州奇迹的许多中心市场与民间工业区都位于非所在行政中心亦非交通要道之地,如永嘉之桥头、温岭之石塘、永康之大园东、古山、苍南之龙港、金乡等等。而这种“大市场中的小角落”恰恰也是苏南乡企的最佳发展基地。苏南最大的行政中心与交通枢纽南京市,市郊诸县的乡企都难以发展。迟至1990年,南京市属地区的乡企在江苏各市中仅居第8位,而无锡、苏州、盐城、南通等就地活跃得多。在这些地区,主要的乡企发展地也往往位于行政辖境的“角落”里。

如丹阳市乡企最发达的皇塘、界牌二镇都是该市最边远的镇,不仅纵贯丹阳境内的铁路、运河、高速公路不达,甚至连丹阳通住各县市的所有干道也不经此二镇。其中,皇塘位于丹阳、金坛、武进三县之交,界牌则位于丹阳、武进、丹徒、扬中四县之交。而城关所在的云阳镇却没有什么乡企。吴江市乡企最发达的盛泽、黎里与震泽三镇也是该县最边远的,位于江浙两省界上,而城关松陵镇乡企产值仅排名第7。宜兴市乡企最发达的太华、丁蜀等镇与全市首富的都山村也是如此,该县最边远的是苏浙皖三省之交、全县最高峰黄塔顶下的太华镇,而该镇乡企产值仅次于丁蜀,外贸供货值则三倍于丁蜀而列全县之首,与此相反,城关所在的宜城镇乡企发展的名次竟在10名以外。

类似现象在苏南十分普遍:诸如金坛市的直溪、指前,太仓市的沙溪,锡山市的前洲、玉祁、查桥等,都是辖境内较偏远乃至最偏远之地,但乡企发展最速,远在城关之上。尽管这些地方城关镇辖区农业人口并不比一般乡镇少,而人均耕地却大都少于一般乡镇,又位于物流、劳务流与信息流的枢纽之地,无论从需要还是从可能看似乎都是最有利于乡企发展的,然而整个苏南地区各县(县级市)中,城关所在镇乡企发达的很少,排名辖境首位的仅张家港市城关杨舍镇与昆山市城关玉山镇二例。而锡山之东亭、武进之湖塘等镇,则是乡企发展、市镇兴旺后县级行政中心才迁入而成为新城关的。

苏南乡企从形式看不同于温州的私营乡企,当地宗族关系也远不如浙、闽、粤诸省发达,但就是这样的“集体企业”为什么也只能在“角落”里发展起来呢?显然,这是因为大一统的强控制不仅压抑了个人活力,也压抑了小共同体的活力。而在这种控制的夹缝中发展起来的“小共同体”自然会带有浓厚的乡土人际关系纽带。它既非“现代企业制度”下的公民契约性经济组织,也非“闾里什伍”的传统行政安排与政社合一的“集体”。苏南乡企自然不是宗族经济,但却往往带有浓厚的“庄主经济”色彩。

内地的早期乡企同样存在类似现象,包括著名的天津静海大邱庄在内的许多乡企明星都是从“角落”中成长的。只是到了90年代中期政府政策进一步放开后,新近的内地乡企发展才在城关附近凸显。如陕西省1997年的五大乡企明星:咸阳秦都区留印村、宝鸡县虢镇、耀县孙原村、临潼骊山镇西街村、歧山县歧星村都是在城关辖区发展起来的。但与苏南不同的是:这些新的乡企活跃区大多以私营经济为主。而与此同时苏南乡企也出现了迅猛的私有化之潮(秦晖 1997)。

这表明,城关附近、交通干道上的流通优势与行政控制优势这两因素中,90年代中期以前一直以后一因素更为突出,它使得“农民企业”为回避行政控制宁可牺牲流通优势,在“角落”里求发展。当然这有个条件,即整个大地区经济较发达,市场网络较活跃并足以伸入这些“角落”。否则,像中西部贫困地区那种封闭的“角落”里乡企也是难以发展的。

这一特征表明:90年代中期以前乡企发展的最重要“优势”,与其说是最能利用市场机制,不如说是最能摆脱行政控制,无论私营型(如温州)还是集体型(如苏南)乡企,在这一点上似乎是共同的。而在不发达与发达地区这一点也基本相似,以下二例可见一斑:

广西桂平县上国村莫兆欣,以四、五人几百元起家搞药厂,曾因违犯医药法规被罚款数万元,上过法庭被告席。后来他屡挫屡起,越搞越大,产值达2亿之巨,成为桂平县的龙头企业。但仍地处桂平大山中最偏僻之地。县里屡屡催他迁厂到城里,并许以种种优惠,但他坚执不肯,仍在本村、乡滚动发展,围绕药厂搞起了纸箱、制瓶、包装、运输、洗印诸企业,又投资并贷款建起了年产值500多万的木材防火加工材料厂和年产值上千万的广西最大水禽养殖场等立足本地的资源型企业。莫兆欣在“角落”中把事业越做越火,以‘大山吃药’名扬全国。他的企业吸纳了全乡60\%的“剩余劳动力”,并向本社区的教育、公益事业大量投资。从而不仅赢得了巨大的乡土声望,而且成为1996年农业部表彰的全国优秀乡镇企业家。

河南新乡小冀镇中街村村民杜天贞以3个人办封头厂,1990年时资产已达1200万元。此时他自愿把厂上交集体,引起轰动,传媒曾以“一个千万富翁的消失与一个富村的产生”为题大炒过一阵。之后他继续当厂长,又任村委会主任(村长),然后入党,“被村民(?)选为村支部书记”,成为村企合一、党政合一的庄主。但他主要的依靠仍是自己的家人与亲族:其高中毕业的妻子是他的“好参谋”,杜家兄弟数人常在家开会帮他决策,尤其是大哥杜天学长期任村干部,老谋深算,对他帮助极大。杜天贞自谓本人仅上过四年小学,靠全家人帮忙才有今天。

这两个案例一南一北,一私一公,但都有许多共同点:依托地缘、血缘共同体,不脱离乡土人际关系与家族的基盘。莫兆欣办的是私营企业,但那种死守山乡不进城的做法却迥异于资本主义“经济人”。杜天贞化私为公,但其企业却更像个家族公司而不像传统的“公社企业”。实际上江浙等发达地区乡企经济中也有类似情形。无论姓私姓公,以“能人”为中心,以小共同体为依托的“庄主经济”都是乡镇企业的通行模式。出现这种状况是不难理解的:既然这时乡企发展最重要的优势不在于它最能利用市场机制,而在于它最能摆脱国家计划控制,那么是否采用最适应于市场关系的产权明晰的私有企业形式便不是最重要的,而对于摆脱大共同体本位的控制来说,一盘散沙式的“伪个人主义”有时反不如自治性小共同体更为有效。

但这种“庄主经济”式的小共同体本位与某些论者宣传的“新集体主义”绝不是一回事。集体主义不管“新”“旧”,都与个体主义构成相对;而“小共同体本位”却是与“大共同体本位”构成相对的。关键在于:改革前的人民公社体制严格地讲并非所谓集体经济,当时公社的生产计划、产品处置、要素配置乃至领导人的任命,都是由政府而非农民“集体”决定的。因此公社经济与国有企业一样是“国家本位”经济,区别只在于国家控制国营企业,国家也承担了控制的后果,(即企业不自主经营,但也不自负盈亏,国家保障工人的收入)而人民公社则是由国家控制但却由农民来承担控制后果的。乡镇企业摆脱了这种状况,即便像南街村这样的“正统集体经济”,也号称“外圆内方”,即对外保持企业法人在市场上的自由经营者地位,脱离了大共同体的控制。这是乡企能创“奇迹”的根本原因。至于“新集体主义”,如果泛指任何形式的联合与协作,那一切现代经济都是这样,算不得乡企的特点;如果是指“公有制”,则“温州”型的乡企固然算不上,就是苏南式的乡企,经过1996年以来的“大转制”与产权改革后,又还有多少这种“主义”的色彩呢?

我国一些乡企的家族色彩、“庄主”色彩与小共同体色彩曾令一些外国人难于理解。在他们看来,这种不符合“现代企业制度”的村社式、在某些情况下甚至近乎农奴制式的体制怎么会有如此活力?而我们的一些论者则把这种状况称为宝贵的传统与“本土文化资源”,甚至称为“超越西方现代性”的一种救世模式。的确,如果不考虑大共同体本位问题,是无法理解这种“小共同体活力”的。但如果想到法家式或拜占廷式的“伪个人主义”之反公民社会性质,那就可以设想,这正是一种“公民与小共同体联盟”以走出传统社会的过渡形态。在这个意义上它与其说是传统的,不如说是反传统(反国家本位之传统)的。它与西方借助王权走向近代固然是不同的途径,但同样要以公民社会为归宿。在这一点上是谈不上什么“超越”的。

除经济外,“公民与小共同体联盟”现象也体现在政治、文化等领域。我国近年来搞“农村基层民主”,在许多地方都受到了“助长宗族势力“的指责。如去年浙江台州前所、杜桥等镇的一些村由农民选出的村委会被镇里撤销,并由政府指派了“村管会”。有趣的是镇里对这种“竞选”不是指责为“资产阶级自由化”而是指责为“宗族作怪”,而由政府任命的村管会头头却往往派的是“资产阶级”,即当地号称“首富”的私营企业家!其实考诸历史,朝廷派遣大私商来推行垄断与统制倒是有传统的,如汉武帝之用桑弘羊、孔仅、东郭咸阳来推行盐铁官营就是一例。 私商的唯利是图、六亲不认在这里并未成为“市民”性格,而是成了大共同体本位的工具。反而是宣扬宗法伦理的儒生(贤良文学)成了民营经济的捍卫者!“伪个人主义”与小共同体在中国传统中的角色于此可见。

与此相类的另一种现象是:我国目前由政府推动的“村级民主”往往都在市场经济很不发达的相对贫困地区进展顺利,如辽宁、河北等地区。据说我国农村第一个村民直选的村委会就出在广西最贫困的河池地区之宜山县(1980年)。而在一些贫困地区,早在改革前旧体制下,由于一穷二白的“集体”没有什么资源可供争夺,因此那时就十分“民主”,生产队长都是轮流当,更无所谓庄主现象。这些地方传统社区组织几为空白,改革后生产队取消,村政(指自然村而非行政村)就几乎不存在了(秦晖:《“村”兮归来》,《中国改革报》1998年5月29日)。除了大共同体本位下官府的厉害外,村民在社区内其实没感到什么压迫。在这些地方,农民的关切点与其说是“社区民主”,不如说是社区自治;而社区民主的含义与其说是限制“庄主”权力,勿宁说主要在于限制政府权力(包括作为“国家经纪”的庄主权力)。因此仅仅把社区民主局限于“民选村官”是远远不够的,重要的问题在于限制国家经纪权而使“村官”更多地体现社区立场,使“村官”能在国家面前维护村人的公民权益。如果反过来,只从国家本位的立场为了削平尾大不掉的庄主,维护大一统价值而搞“村级民主”,像改革前以往常用“运动民主”来加强一元化体制那样,那就意义不大。

而在东南诸省市场经济发达的富裕农村,这些年来许多地方村政的演变不是表现为“民选村官”,而是表现为村企合一、企业“吃掉”村级组织、“村子公司化,支书老板化”、“庄主经济”演变为“庄主政治”。而企业的“一长制”则演变为社区的“一主制”。如果不考虑大共同体本位体制的解构问题,这样的演进就几乎是一种“反动”的现代领主制。而像“禹作敏现象”这类“庄主制”之弊也在知识界引起了广泛批评。然而人们却很少从传统中国社会向公民社会演进的角度对“庄主现象”作出深刻的反思。

实际上传统中国不同于小共同体本位的西方,除了帝国解体的特殊时期(如魏晋时期)外,很难出现真正意义上的“领主”之弊。中国历史上的“庄主”,要么以“国家经纪”身分在官府支持下为弊,类似于《水浒》中的祝家庄、曾头市之“庄主”那样。这种形式的“庄主之弊”实质上与吏治腐败一样是大共同体本位之弊,并不是单纯的“庄主”问题。要么“庄主”作为一种可能制衡全能国家的自治力量起到“保护型经纪”作用,这种庄主自然也会生弊,但比起全能国家之弊、官府胥吏腐败专横之弊来却是次要的。因而我国历史上屡见农民宁当“私属”而逃避为“编氓”的现象,甚至“庄客”支持“庄主”抗官的现象。所以在中国批判“庄主”现象有个从公民权利出发还是从全能国家权力出发的问题。改革时代东南地区的“庄主政治”当然谈不上是“中国传统”超越了“西方民主”,但比改革前“一元化”控制下许多赤贫农村“干部轮流当”式的“民主”还是一种进步。

在文化上,“公民与小共同体的联盟”在改革中的农村也留下了痕迹。近年来学界对东南农村中的修谱造祠之潮甚为关注,但我觉得更值一提的是浙江等一些农村地区的“村志”现象。如永康县清溪乡(今永康市清溪镇)1986年由乡文化站文化员、乡初中卸任教务长等人在乡政府支持下倡修《清溪乡志》,消息一出,乡属各村纷纷响应,结果在《乡志》编成前已有八九部村志先期而出。这些民间编印的“村志”宗谱色彩浓厚。如《官川村志》356页中就有314页即全书篇幅的88\%是宗谱,编者明言是“借编志东风,重修家乘”(《永康官川村志》1987年自印本宗谱页第四)。但既为一村之志,所以又与传统宗谱不同,除了本村主姓之外,还记载了其他姓氏村民的谱系。如《官川村志》中除了该村主姓的《胡氏宗谱》外,还有《官川其他姓氏支流世系》和《官川村各姓氏传列》。从80年代后期到90年代中期,在这种“村志现象”的发展中,不断受到个性解放的公民文化从一个方向、坚持一元化控制的大共同体本位文化从另一个方向的双重影响。如后于《官川村志》而出的《山西村志》的宗谱初稿依传统只列男系,付排之后便有村民提议:“在世系排列上应与宗谱有所不同,女的要求排上”,迫使该志抽版重排(《山西村志》,1988年自印本)。到了1994年的《河头村志》,便出现了《村民世系表》这种形式,它把本村村民从主姓吕氏直到只有一户人的贾氏,从明初最早定居河头,迄今已传23代的吕家直到公社化时代才入居该村的戴、潘等姓,不分男女,人人入谱。而且各姓氏不分大小一律以始居河头者为世系之源,废止了传统族谱乱攀远祖以显其贵的陋习。

另一方面,农民们对来自大共同体的“禁谱令”进行了巧妙的抵制。《河头村志》在寻求正式出版时,出版社根据禁止出版族谱类书籍的有关规定不让收入《世系表》,农民们便来了个移花接木。结果问世的村志虽印数仅1500册,却有两个“版本”:交出版社发行的没有《世系表》,据说只印了100本以蒙混上头,而其余1400本都在装订时暗中加上了《世系表》,由河头村自己发行与赠送。闻此内情者莫不感叹!

从1987年装订简陋的自印本《官川村志》到1994年以来铜版精装正式出版的《河头村志》及以后的《前洪村志》《雅庄村志》等,从“宗谱”到“村民世系表”,我们看到当地农民的“小共同体意识”在明显增强,而这显然不是以扼杀公民个性与个人权利意识为代价的。

\currentpdfbookmark{小共同体与公民社会的前途――兼论“新”儒家如何可能}{future}
\section*{小共同体与公民社会的前途――兼论“新”儒家如何可能}

总而言之,改革时代中国除了公民意识在成长(尤其在城市中)外,最显著的变化是小共同体在经济、政治与文化各层面的凸显(尤其在农村)。这些小共同体由于微观上带有传统色彩(例如宗族色彩)而引起了两种议论:或斥其为“封建复辟”,或褒之为“传统活力”。然而从“公民与小共同体联盟”的角度看,这种小共同体的兴起与西方历史上民族国家的兴起一样是有正面意义的。另一方面,这种“宗族的崛起”与西方历史上“王权的崛起”一样只是走向个人本位现代社会途中的阶段性现象。现代化的完成终会消除宗族束缚或其他类型的“小共同体束缚”,正如其在西方消除了王权的束缚一样。当然,这并不排除宗族作为一种非强制性的象征符号继续存在,正如在西方王室可以作为象征符号继续存在一样。

这自然只是一种可能性。如今断言中国传统社会会经由“公民与小共同体的联盟”,成功地走完西方社会经由“公民与大共同体的联盟”为中介而实现的现代化过程,还为时过早。因为即使在西方历史上,“公民与王权的联盟”也并不必然会导致现代化。从逻辑上讲,这种“联盟”既有可能使“公民”利用“王权”战胜领主而走向现代化,但也有可能使“王权”利用“公民”战胜领主,而出现传统中央集权专制帝国。如16世纪的西班牙就是如此。在那里王权的胜利和领主的没落并没有使采邑制的废圩上生成公民社会的基础结构,因而公民也就不可能在此后依托这种结构战胜王权。结果,查理五世与腓力二世的集权帝国虽然一度称雄于世,却演变为一个极端保守的老大帝国,并使西班牙长期成为欧洲病夫与落伍者。

同样,“公民与小共同体联盟”的结果既可能是前者利用后者,也可能是后者利用前者。在后一情况下,“庄主”现象发展为诸侯现象,一元化体制的解体不是导致公民社会而是导致传统的乱世,“朝廷”的“自由放任”没有放出一个“中产阶级”,却放出无数土皇帝与土围子。在这种情况下中央集权会变成领主林立,统一国家会变为一盘散沙,但这只是历史上“合久必分”的重演而并无现代化意义。当然,这种情况也可能导致另一种结局,即由于“庄主”过份的负面影响使“公民”重新寻求大共同体的庇护,造成传统一元化体制的复归。这正如西方“公民与王权的联盟”除了可能导致“西班牙现象”外也可能导致“意大利现象”,即由于王权过份的负面影响使公民重新寻求小共同体庇护,造成近代南意大利与西西里的黑手党式帮派社会一样。

因此无论在中国还是在西方,现代化与公民社会的实现都不是“必然”的,历史决定论的解释完全错误。“公民”无论与大共同体还是与小共同体结盟都是有风险的。但现代化作为一种价值则对传统中国人与传统西方人同具吸引力,文化决定论的解释同样是错误的。

改革20年来随着小共同体的不断成长,“庄主”现象的负面作用已引起人们注意,但究竟是以法家的思路还是以现代法治的思路解决这一问题,是以“王权”还是以公民权来制约“庄主”,则事关重大。在此我们不妨再做个历史比较:在西方历史上“公民与王权的联盟”要避免公民被王权所利用,就要走社会改造先行之路,即先在“传统”王权之下变小共同体本位社会为公民社会,然后再以公民社会组织为纽带制衡王权(一盘散沙式的人们是不可能制衡王权的),变王朝国家为公民国家。显然,能否先形成公民社会便成为问题的关键。

那么,在“公民与小共同本联盟”的条件下,逻辑上就要求走国家改造先行之路,即先在“传统”小共同体之上变传统国家为公民国家(民主国家),然后再以民主国家为依托制约“庄主”(无政府状态下的人们是无法对付“庄主”的),变小共同体本位为公民社会组织。显然,在这种情况下能否先形成公民国家便成为问题的关键。

以上两种情况可以简表如下:
\begin{enumerate}
\item 西方现代化
  \begin{figure}[H]
    \centering
    \makebox[\textwidth][c]{\begin{tikzpicture}[font=\small]
      \node[anchor=west,align=left] at (0,0) {公民与大共同体的联盟};
      \node[anchor=west,align=left] at (0,-.5) {公民社会(契约)};
      \node[anchor=west,align=left] at (0,-1) {传统民族国家(专制)};
      \ifafourpaper
      \node[anchor=west,align=left] at (-3.6,0) {小共同体本位};
      \node[anchor=west,align=left] at (-3.6,-.5) {传统社会(依附)};
      \node[anchor=west,align=left] at (-3.6,-1) {“前国家”(伪民主)};
      \draw (-1.1,0) -- (.1,0);
      \draw (4.05,0) -- (4.66,0);
      \node[anchor=west,align=left] at (4.56,0) {公民(个人)本位};
      \node[anchor=west,align=left] at (4.56,-.5) {公民社会(契约)};
      \node[anchor=west,align=left] at (4.56,-1) {公民国家(民主)};
      \else
      \node[anchor=west,align=left] at (-3,0) {小共同体本位};
      \node[anchor=west,align=left] at (-3,-.5) {传统社会(依附)};
      \node[anchor=west,align=left] at (-3,-1) {“前国家”(伪民主)};
      \draw (-.92,0) -- (.1,0);
      \draw (3.33,0) -- (3.88,0);
      \node[anchor=west,align=left] at (3.8,0) {公民(个人)本位};
      \node[anchor=west,align=left] at (3.8,-.5) {公民社会(契约)};
      \node[anchor=west,align=left] at (3.8,-1) {公民国家(民主)};
      \fi
    \end{tikzpicture}}
  \end{figure}
\item 中国现代化
  \begin{figure}[H]
    \centering
    \makebox[\textwidth][c]{\begin{tikzpicture}[font=\small]
      \node[anchor=west,align=left] at (0,0) {公民与小共同体的联盟};
      \node[anchor=west,align=left] at (0,-.5) {传统社会(依附)};
      \node[anchor=west,align=left] at (0,-1) {公民国家(民主)};
      \ifafourpaper
      \node[anchor=west,align=left] at (-4.14,0) {大共同体本位};
      \node[anchor=west,align=left] at (-4.14,-.5) {“前社会”(伪个人主义)};
      \node[anchor=west,align=left] at (-4.14,-1) {传统民族国家(专制)};
      \draw (-1.65,0) -- (.1,0);
      \draw (4.05,0) -- (4.66,0);
      \node[anchor=west,align=left] at (4.56,0) {公民(个人)本位};
      \node[anchor=west,align=left] at (4.56,-.5) {公民社会(契约)};
      \node[anchor=west,align=left] at (4.56,-1) {公民国家(民主)};
      \else
      \node[anchor=west,align=left] at (-3.45,0) {大共同体本位};
      \node[anchor=west,align=left] at (-3.45,-.5) {“前社会”(伪个人主义)};
      \node[anchor=west,align=left] at (-3.45,-1) {传统民族国家(专制)};
      \draw (-1.37,0) -- (.1,0);
      \draw (3.33,0) -- (3.88,0);
      \node[anchor=west,align=left] at (3.8,0) {公民(个人)本位};
      \node[anchor=west,align=left] at (3.8,-.5) {公民社会(契约)};
      \node[anchor=west,align=left] at (3.8,-1) {公民国家(民主)};
      \fi
    \end{tikzpicture}}
  \end{figure}
\end{enumerate}

因此笔者反对那种粗陋的类比,即从西方历史的发展得出“先有自由后有民主”,因而民主必须缓行的“规律”论。的确,就西方而言“王权庇护下的自由”是“公民与王权的联盟”现代化之路的重要阶段。但中国走的是另一条道路,清末、民国的历史已表明,中国式的王权是无法保护“自由”的。而中国式的小共同体能否有助于“民主”,则有待于我们的实践。

在“公民与小共同体联盟”的道路上,文化领域的一大问题就是所谓的“新”儒家复兴是否可能。这里我们说的不是儒家的个别词语如“天下为公”“民贵君轻”“因民之所利而利之”之类能否与现代价值相通,而是就儒家的基本价值而言。在儒家的基本价值体系中缺乏个人本位的公民权利观念,而没有这个就谈不上现代公民社会、市场经济与民主政治,谈不上由身份社会向契约社会的过渡。因此儒学不能取代“西学”。至于说儒学能否“超越现代性”而给人类、包括西方在内指出一条通往“后现代”之路,那就更为虚玄,本文无法在此评述。

问题是:在“公民与小共同体联盟”的中介状态下,儒家思想资源的意义何在?这就涉及我们对真正的(而非典籍上的)“中国传统”的理解。如前所述,中国的真正传统是“儒表法里”,而表里之间虽经董仲舒以来两千多年的改造,仍然是有矛盾的。文革时期的“批儒弘法”与“马克思加秦始皇”之论虽然充满了附会、影射及“古为今用”的曲解,却不能仅仅视之为一大历史玩笑。 大共同体本位体制与儒家价值的矛盾,从秦汉以来的确是一直存在的。

儒家无个人本位之说,但却有“共同体多元化”倾向而反对大共同体一元化。《孟子・离娄上》说:“人各亲其亲,长其长,而天下平。”而法家则坚决反对亲亲之说。《商君书・开塞》云:“亲亲则别,爱私则险,民众以别险为务,则民乱。”于是便有了爹亲娘亲不如领袖亲的价值观,和为大一统的法、术、势可以六亲不认的法吏人格。(瞿同祖,1981)

从共同体多元化立场出发,儒家认为每个共同体内都有长幼亲疏贵贱上下之别,不平等是普遍的,“物之不齐,物之情也”(《荀子・荣辱篇》)。这虽有违现代平等价值,却与大一统的“编户”必须“齐民”的观念相矛盾。而法家则鼓吹大共同体本位体制下普遍奴隶制的“平等”,人人为皇上之奴,彼此不得有横向依附,“不知亲疏、远近、贵贱、美恶”,一以度量断之(《管子・任法》)。极而言者甚至把人人都视为皇权之下的“无产者”:“三代子百姓,公私无异财;人主擅操柄,如天持斗魁;赋予皆自我,兼并乃奸回,奸回法有诛,势亦无自来。”(王安石:《兼并》诗) 由此产生了统制经济下的“抑兼并”思想,认为“万民之不治”的原因是“贫富之不齐”(《管子・国蓄》)。因而需要“令贫者富,富者贫”,甚至公然声称要“杀富”(《商君书・说民、弱民》)!而这种禁止人民各有其“私”的皇权管制下的“平均主义”其实是最大的不公,即黄宗羲抨击的“以我之大私为天下之大公”。

如此等等。显然,儒家思想本身不是现代化理论,但它作为一种共同体多元化学说对大共同体本位,尤其对于极端的大共同体一元化体制,是有解构作用的。儒家思想更不是什么“后现代的救世理论”,但它在中国的条件下也并非现代化之敌人。而在“公民与小共同体联盟”的条件下,出现现代公民意识与“儒家传统”的联盟也不是不可能的,这种联盟中的儒家也许就是真正的“新”儒家。但是,这种“新”儒学必须不是以解构所谓“西学”,而是以解构中国法家传统为已任的。“新”儒学的对立面不是公民权利,而是大共同体独尊。这就要求“新”儒学理论必须公民本位化,而不是国家主义化。否则儒学就无法跳出董仲舒以来儒表法里的怪圈,它的前途也就十分可疑。

\begin{thebibliography}{99}
\bibitem[杜赞奇(1994)]{Duara} 杜赞奇(Prasenjit Duara). 1994. 文化、权力与国家:1900—1942年的华北农村. 中译本. 江苏人民出版社.
\bibitem[杜正胜(1990)]{Du} 杜正胜. 1990. 编户齐民:传统政治社会结构之形成. 台北联经出版公司.
\bibitem[费成康(1998)]{Fei} 费成康 主编. 1998. 中国的家法族规. 上海社会科学院出版社.
\bibitem[甘阳(1994)]{Gan} 甘阳. 1994. 乡土中国的重建. 香港二十一世纪,总第26期.
\bibitem[格罗索(1994)]{Grosso} 格罗索(Grosso, G). 1994. 罗马法史. 中译本. 中国政法大学出版社.
\bibitem[侯宜杰(1993)]{Hou} 侯宜杰. 1993. 二十世纪初中国政治改革风潮. 人民出版社.
\bibitem[黄永豪(1987)]{Huang} 黄永豪. 1987. 清代珠江三角洲的沙田、乡绅、宗族与租佃关系. 香港中文大学硕士论文. 未刊手稿.
\bibitem[金雁 等(1996)]{Jin} 金雁,卞悟. 1996. 农村公社. 改革与革命. 中央编译出版社.
\bibitem[堀敏一(1996)]{Hori} 堀敏一. 1996. 中国古代の家と集落. 东京汲古书院.
\bibitem[梁漱溟(1990)]{Liang} 梁漱溟. 1990. 乡村建设理论. 载梁漱溟全集第二卷. 山东人民出版社.
\bibitem[楼劲 等(1992)]{Lou} 楼劲,刘光华. 1992. 中国古代文官制度. 甘肃人民出版社.
\bibitem[秦晖(1986)]{Qin86} 秦晖. 1986. 甲申前后北方平民地主的政治动向. 陕西师大学报,1986年第3期.
\bibitem[秦晖(1987)]{Qin87} 秦晖. 1987. 郫县残碑与汉代蜀地农村社会. 陕西师大学报,1987年第2期.
\bibitem[秦晖(1995)]{Qin95} 秦晖. 1995. 宗族文化与个性解放:农村改革中的宗族复兴与历史上的宗族之谜. 东京中国研究,总第8期.
\bibitem[秦晖 等(1996)]{Qin96} 秦晖,苏文. 1996. 田园诗与狂想曲:关中模式与前近代社会的再认识. 中央编译出版社.
\bibitem[秦晖(1997)]{Qin97} 秦晖. 1997. 江浙乡镇企业转制案例研究. 香港中文大学.
\bibitem[秦晖(1998)]{Qin98} 秦晖. 1998. 公社之谜. 香港二十一世纪,总第48期.
\bibitem[黄永豪(1981)]{Qu} 瞿同祖. 1981. 中国法律与中国社会. 中华书局.
\bibitem[沈志华(1994)]{Shen} 沈志华. 1994. 新经济政策与苏联农业社会化道路. 中国社会科学出版社.
\bibitem[盛洪(1999)]{Sheng} 盛洪. 1999. 为万世开太平. 北京大学出版社.
\bibitem[韦伯(1995)]{Weber} 韦伯, M. 1995. 新教伦理与资本主义精神. 三联书店.
\bibitem[俞伟超(1988)]{Yu} 俞伟超. 1988. 中国古代公社组织的考察:论先秦两汉的“单—僤—弹”. 文物出版社.
\bibitem[张乐天(1998)]{Zhang} 张乐天. 1998. 告别理想:人民公社制度研究. 上海东方出版中心.
\bibitem[郑振满(1992)]{Zheng} 郑振满. 1992. 明清福建家族组织与社会变迁. 湖南教育出版社.
\bibitem[朱苏力(1998)]{Zhu} 朱苏力. 1998. 法制及其本土资源. 中国政法大学出版社.
\bibitem[Bloch(1962)]{Bloch} Bloch, M. 1962. \textit{Feudal Society}. 2 vols. London.
\bibitem[Furnivall(1944)]{Furnivall} Furnivall, J. S. 1944. \textit{Netherlands India: A Study of Plural Economy}. Cambridge University
\bibitem[Freedman(1966)]{Freedman} Freedman, M. 1966. \textit{Chinese Lineage and Society: Fukien and Kuangtung}. University of London.
\bibitem[Gruhbos(1995)]{Gruhbos} Gruhbos, J. E. 1995. \textit{Law and Family in late Antiquity: the Emperor Constantine’s Marriage Legislation}. Oxford.
\bibitem[James(1974)]{James} James, M. 1974. \textit{Family, Lineage and Civil Society: A Study of Society, Politics and Mentality in the Durham Region, 1500–1640}. Oxford.
\bibitem[Marriot(1955)]{Marriot} Marriot, M., ed. 1955. \textit{Village India: Studies in the Little Community}. University of Chicago.
\bibitem[Oborensky(1971)]{Oborensky} Oborensky, D. 1971. \textit{The Byzantine Commonwealth, Eastern Europe, 500–1453}. London.
\bibitem[Popkin(1979)]{Popkin} Popkin, S. 1979. \textit{The Rational Peasant, the Political Economy of Rural Society in Vietnam}. Berkeley.
\bibitem[Redfield(1955)]{Redfield55} Redfield, R. 1955. \textit{The Little Community}. University of Chicago.
\bibitem[Redfield(1956)]{Redfield56} \entimes———. 1956. \textit{Peasant Society and Culture}. University of Chicago.
\bibitem[Scott(1976)]{Scott} Scott, J. C. 1976. \textit{The Moral Economy of the Peasant: Rebellion and Subsistence in Southeast Asia}. New Haven.
\bibitem[Treadgold(1997)]{Treadgold} Treadgold, W. 1997. \textit{A History of the Byzantine, State and Society}. Stanford University.
\end{thebibliography}
\end{document}

% Local Variables:
% TeX-engine: luatex
% End:
